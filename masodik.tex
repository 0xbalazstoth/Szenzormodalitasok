\section{Téma 2}

\subsection{Szenzorjelek feldolgozása}
\begin{itemize}
    \item Miért
    \begin{itemize}
        \item Az adatokat értelmezni kell (például: lényeges információ a megfigyelt személy egészségi állapotáról, helyzet tudatosság, \dots)
        \item Adott paramétertér állapotváltozásának nyomonkövetése (trendek megfigyelése)
        \item Amit nem ismerünk, azt nem tudjuk mérni, amit tudunk mérni, azt meg tudjuk ismerni.
    \end{itemize}
    \item Probléma
    \begin{itemize}
        \item Sok adat jön, redundáns az információ
        \item Jönnek hibás, téves adatok
        \item A nyilvánvaló adatok feldolgozása "egyszerű" (például: nincs szívhang, 42C-os hőmérséklet), és a többi szenzorjel közötti korreláció is érdekes
        \item A sok adatot nehéz visszakereshetően tárolni
    \end{itemize}
\end{itemize}

\clearpage
\subsection{Élettani paraméterek (típusok)}
\subsubsection{Organizmus mérhető paraméterei}
\begin{itemize}
    \item Fiziológiai paraméterek (fizikai, kémiai, \dots)
    \item Belül mért (invazív)
    \begin{itemize}
        \item Folyadékok (vér), minták, \dots
    \end{itemize}
    \item Kívül mért (nem invazív)
    \begin{itemize}
        \item Vérnyomás, pulzus, EKG, EEG, EMG, \dots
        \item Bőrhőmérséklet, szín, \dots
    \end{itemize}
    \item Mikro/makró környezet
    \item Páratartalom, hőmérséklet, tartás, gyorsulás, sebesség, \dots
\end{itemize}

\subsection{Szenzoradat kezelés}
\begin{itemize}
    \item Data Acquisition (DAQ) (egy/több szenzor)
    \item Adatkezelés (feldolgozás/szűrés)
    \item Tárolás, keresés
    \item Vizualizáció
    \item Megosztás
\end{itemize}

\subsection{Mért élettani paraméterek}
\begin{itemize}
    \item Tudományos munka általában:
    \begin{itemize}
        \item Mérés tervezése
        \item Mérés kivitelezése ismert környezetben
        \item Adatok előfeldolgozása (tisztítás, szűrés, \dots)
        \item Adatok feldolgozása
        \item Adatok kiértékelése
        \item Döntések
    \end{itemize}
\end{itemize}

\subsection{Mérési hibák}
\begin{quote}
    Minden mérés tartalmaz hibákat!
\end{quote}
\begin{itemize}
    \item Hibák jönnek:
    \begin{itemize}
        \item Érzékelési hibák
        \begin{itemize}
            \item A $\rightarrow$ D konverzió (kvantifikálás)
            \item Mintavételi hibák (Nyquist-Shannon mintavételi elmélet)
        \end{itemize}
        \item Mérési környezet/elrendezés
        \item Mérőműszer problémák
        \begin{itemize}
            \item AAMI - American Association for the Advancement of Medical Instrumentation, BHS-British Hypertension Society (A osztály 40\%: 5Hgmm)
        \end{itemize}
    \end{itemize}
    \item Távoli mérések is tartalmaznak hibákat.
\end{itemize}

\subsection{Távoli szenzor adat}
\begin{itemize}
    \item A szenzorokból jövő adatok egy adott paraméter (hőmérséklet, hely, pulzusszám, \dots) digitalizált értékét jelentik.
    \item Ezt valamilyen mintavételi frekvenciával mért mintákból kapjuk meg, amik többnyire átlagok.
    \item A mérések gyakorisága fontos jellemzője a rendszernek, befolyásolja a végezhető adatelemzés felbontoképességét.
    \item Idő kell mire beérkezik
    \begin{itemize}
        \item Valós idő
        \item Szakaszos küldés $\rightarrow$ Mit tárolunk?
    \end{itemize}
\end{itemize}

\subsection{Egyetlen vs. több szenzoros mérések}
\begin{itemize}
    \item Több szenzor $\rightarrow$ több mért érték (és mért adat)
    \item Kérdés $\rightarrow$ Melyik érték pontos(abb)/igaz(abb)?
    \item Mit használjunk?
    \begin{itemize}
        \item Átlag? Súlyozott Átlag? \dots
    \end{itemize}
\end{itemize}

\subsection{Mérnöki kihívások a nagymennyiségű adatgyűjtésnél}
\begin{itemize}
    \item Általános kihívások
    \begin{itemize}
        \item P1 $\rightarrow$ Sok DAQ csomópont
        \item P2 $\rightarrow$ Sok szenzor (különböző típusú)
        \item P3 $\rightarrow$ Nagy adatmennyiség lehetőleg gyorsan átküldve
    \end{itemize}
    \item Kommunikációs probléma $\rightarrow$ P1 $\times$ P2 $\times$ P3
    \item Szoftver kihívások
    \begin{itemize}
        \item Valós idejű DAQ + előfeldolgozás + feldolgozás + megjelenítés (különböző tartományok)
        \item Komplex döntési helyzetek
        \item Online adatmenedzsment (megosztás + archiválás)
    \end{itemize}
    \item Hardver korlátok
    \begin{itemize}
        \item Energia problémák
        \item Kommunikációs hatótávolságok, adat multiplexálási problémák/idő, frekvencia
        \item Biztonság, megbízhatóság, használhatóság
    \end{itemize}
\end{itemize}

\subsection{Biológiai jellemzők és mérésük}
\begin{itemize}
    \item Vérnyomás
    \item Vér alkotó elemeinek mérése $\rightarrow$ vércukor, koleszterin (LDL, HDL), laktát, hemoglobin, triglicerid
    \item Légzési paraméterek
    \item EKG, EEG, EMG, ECOG
    \item Testhőmérséklet
    \item GSR
    \item Súly, mozgásmennyiség
\end{itemize}

\subsection{Vérnyomás}
\begin{itemize}
    \item A magas vérnyomás népbetegség
    \item A betegek száma folyamatosan nő
    \item A vérnyomáscsökkentő gyógyszerek piaca hatalmas
    \item A kezelés módja erősen függne a mért értékektől (legkisebb dózis)
    \item A vérnyomás viszonylag gyorsan jelentősen változó érték (a szabályozó rendszer instabil/nem robosztus)
\end{itemize}

\subsection{Vérnyomásmérés}
\begin{itemize}
    \item Az egyik legtöbbet mért fiziolóiai paraméter
    \item Több millió eladott berendezés Európában évente
    \item A készülékek mérési pontossága erősen eltérő
    \item Csak tájékoztató jellegű információt ad
    \item Sok mérési megoldásnál megfigyelhető a szervezet autoregulációs hatása $\rightarrow$ A mérés beavatkozik a keringési rendszer biomechanikájába
\end{itemize}

\subsection{Humán keringési rendszer}
\begin{center}
    \customwidthimage{fejf}{13cm}
\end{center}

\subsection{Szívverés - szisztolé és diasztolé}
\begin{itemize}
    \item Minden szívverés két fázisból áll, ezeket az orvosok szisztoléként és diasztoléként határozzák meg.
    \item A szisztoléban a szívizom összehúzódik és vért pumpál a keringésbe, míg a diasztolé alatt ellazul és újratöltődik vérrel.
    \item Vérnyomás ingadozás okai
    \begin{itemize}
        \item Hosszú idejű variabilitás (napi bioritmus) periódikus $\rightarrow$ +-20-40 Hgmm változás (24 órás ciklus)
        \item Rövid idejű változékonyság, néhány perces hatások
        \begin{itemize}
            \item Légzés, fizikai aktivitás, drogok/koffein/tea, +-20 Hgmm
            \item Pszichés hatások, fehér köpeny effektus +30Hgmm
        \end{itemize}
    \end{itemize}
\end{itemize}

\subsection{Vérnyomásértékek}
\begin{table}[h!]
    \centering
    \footnotesize
    \begin{tabularx}{\textwidth}{|X|X|X|}
    \hline
    \textbf{Kategória} & \textbf{Szisztolés nyomás (Hgmm)} & \textbf{Diasztolés nyomás (Hgmm)} \\ \hline
    Optimális vérnyomás & < 120 & < 80 \\ \hline
    Normális vérnyomás & 120-129 & 80-84 \\ \hline
    Emelkedett-normális vérnyomás & 130-139 & 85-89 \\ \hline
    Kóros vérnyomás - hipertónia & 140 < & 90 < \\ \hline
    I. fokozat (enyhe hipertónia) & 140-159 & 90-99 \\ \hline
    II. fokozat (középsúlyos) & 160-179 & 100-109 \\ \hline
    III. fokozat (súlyos hipertónia) & >= 180 & >= 110 \\ \hline
    Izolált diasztolés hipertónia & < 140 & > 89 \\ \hline
    Izolált szisztolés hipertónia & >= 140 & < 90 \\ \hline
    \end{tabularx}
    \label{your_label_here}
\end{table}

\subsection{A szívműködés jellemző jelei}
\customwidthimage{sziv}{13cm}

\subsection{Nagyvérköri nyomásértékek}
\customwidthimage{nagyv}{10cm}

\subsection{A keringési rendszer egyszerűsített modellje}
\begin{itemize}
    \item Szív, aorta és a bal artériának illetve vénáinak egyszerűsített villamos helyettesítő képe.
    \item Feszültség = nyomás, az áram pedig a térfogat-, illetve tömegáram.
    \item A kapacitások az aorta és a vénák puffer hatását reprezentálják.
    \item Diódák helyettesítik a billentyűket, transzformátor a kapilláris hálózatot és ellenállások az áramlási ellenállást.
\end{itemize}
\customwidthimage{ker}{13cm}

\begin{itemize}
    \item Áramlási ellenállás tetszőleges érdarabra: \\
    \[ R = \frac{8L\eta}{r^4\pi} \]
    \begin{itemize}
        \item $L$ az ér hossza,
        \item $r$ a belső sugár,
        \item $\eta$ a vér viszkozitása
    \end{itemize}
    \item A nagy vérkör sorosan kapcsolt szakaszainak ellenállás eloszlása:
\end{itemize}
\begin{table}[h!]
    \centering
    \footnotesize
    \begin{tabularx}{\textwidth}{|X|X|}
    \hline
    Aorta, nagy artériák & 10\% \\ \hline
    Kis artériák (prekapilláris sphincterek) & 50-55\% \\ \hline
    Kapillárisok & 30-35\% \\ \hline
    Vénák & 5\% \\ \hline
    \end{tabularx}
    \label{your_label_here}
\end{table}

\subsection{Mérések pontossága}
\begin{itemize}
    \item Készülékek minőségbiztosítása és pontossága, publikáltak szabványokat
    \begin{itemize}
        \item AAMI - American Association for the Advancement of Medical Instrumentation
        \item BHS - British Hypertension Society
        \item Higanyos manométer az etalon
        \begin{itemize}
            \item A osztály: mérések 40\%-ában 5Hgmm, 15\%-ában 10Hgmm, 5\%-ában 15Hgmm eltérés
        \end{itemize}
    \end{itemize}
\end{itemize}

\clearpage
\subsection{Mérési megoldások/elvek (nem invazív)}
\begin{itemize}
    \item Auszkultációs módszer
    \begin{itemize}
        \item Korotkov hangok detektálásán alapul
        \item Bal felkarra mandzsettás mérő, felfújjuk, alatta sztetoszkóppal figyeljük a szívverést
        \begin{itemize}
            \item Pulzálás megszűnik $\rightarrow$ Szisztolés nyomás
            \item Leeresztjük a mandzsettát $\rightarrow$ Turbulens áramlások (Korotkov hangok)
            \item Amikor megszűnnek $\rightarrow$ Diasztolés nyomás
        \end{itemize}
        \item \textbf{Hátrány} $\rightarrow$ Egyetlen pillanatnyi érték, mandzsetta befolyásol, leeresztési sebesség befolyásol, szubjektív a hallgatóság
    \end{itemize}
    \item Oszcillometriás módszer
    \begin{itemize}
        \item Automata vérnyomásmérők
        \item Marey fedezte fel
        \item Az artéria pulzálása megjelenik a felkarra helyezett mandzsetta nyomásában
        \item A pulzálás maximális $\rightarrow$ Artériás középnyomással (MAP)
        \item A szisztolés és diasztolés érték ebből számolható definiált szorzókkal
        \item \textbf{Hátrány} $\rightarrow$ A szorzók pontatlanok, öregedés (artériák rugalmatlanok)
    \end{itemize}
\end{itemize}

\clearpage
\subsection{Oszcillometriás vérnyomásmérő elvi felépítése}
\begin{center}
    \customwidthimage{oszc}{10cm}
\end{center}

\subsubsection{A nyomásmérő rész blokkvázlata (oszcillometriás)}
\begin{center}
    \customwidthimage{oszc2}{10cm}
\end{center}

\subsubsection{A nyomásmérő rész aluláteresztős erősítőjének megvalósítása}
\begin{center}
    \customwidthimage{oszc3}{10cm}
\end{center}

\subsection{Mérési megoldások/elvek (nem invazív)}
\begin{itemize}
    \item Tonometria
    \begin{itemize}
        \item Legpontosabb nem invazív mérési módszer
        \item Mechanikai tapintófejjel rögzítésre kerül a csuklóartéria pulzálása (folytonos nyomásgörbe)
        \item \textbf{Hátrány} $\rightarrow$ Költséges megoldás
    \end{itemize}
    \item PPG-alapú vérnyomásmérés
    \begin{itemize}
        \item A fotopletizmográf (PPG) a hajszálerek térfogat változását regisztrálja (pl:a bal kéz egyik ujjbegyén)
        \item Mandzsettával mérik + EKG görbe
        \item Bal felkaron mandzsetta $\rightarrow$ Nyomása meghaladja a diasztolés értéket, a PPG hullámok amplitúdója elkezd csökkenni, majd meghaladva a szisztolés értéket eltűnik.
        \item A szisztolés nyomás ezzel nagy pontossággal mérhető (az egyetlen bizonytalanság a felfújás sebességéből adódik)
        \item A diasztolés nyomásnál a PPG amplitúdója modulálja a légzést
        \item Az EKG R-hullám és a PPG pulzus közötti késleltetés méréséből számolható a szisztolés vérnyomás
        \item \textbf{Hátrány} $\rightarrow$ Bonyolult, sok nehezen követhető változó
    \end{itemize}
\end{itemize}
\[ BP = \frac{1}{\alpha} \left[ \ln \left( \frac{L^2 d\rho}{E_0 h} \right) - 2 \ln(\Delta T_{PT}) \right] \]

\subsection{Vérnyomásmérők a gyakorlatban}
\begin{itemize}
    \item Felkaros
    \item Csuklós
    \item Ujjbegyes
\end{itemize}

\subsubsection{ABPM - vérnyomás Holter}
\begin{itemize}
    \item Hosszú idejű mérések
    \item BHS és AAMI validált pontosság
    \item 190g
    \item Felkaros
    \item ABPM report
\end{itemize}

\subsubsection{Vérnyomásmérést befolyásoló tényezők}
\begin{itemize}
    \item Testhelyzet
    \begin{itemize}
        \item Keresztve tett láb + 8Hgmm, ülő helyzet + 5Hgmm, \dots
    \end{itemize}
    \item Mérési technológia
    \item Napszak
    \item Érzelmi állapot
    \item Fizikai aktivitás
    \item Karvastagság, illetve a mandzsetta aránya (méret)
    \begin{itemize}
        \item Karkörfogat 80\%-a
    \end{itemize}
\end{itemize}

\clearpage
\subsubsection{Invazív vérnyomásmérés}
\begin{itemize}
    \item Intravaszkuláris vérnyomás érzékelő
    \item Szűkült érszakasz áramlás és nyomásviszonyairól ad információkat
    \item Katéteres nyomásmérési megoldások
    \begin{itemize}
        \item Mikromanométer végű vezető drót végén nyomásérzékelő
        \item Fiziológiás folyadékkal feltöltött katéter (érzékelő a külső szerelékben van)        
        \item Katéteres mikro optikai nyomásérzékelő (MOMS)
    \end{itemize}
\end{itemize}
\customwidthimage{kat}{12cm}

\subsubsection{Vér összetevők mérése}
\begin{itemize}
    \item Vércukormérés
    \item Koleszterinszint mérés
    \begin{itemize}
        \item LDL-koleszterin mérés
        \item HDL koleszterin mérés
        \item VLDL koleszterin mérés
    \end{itemize}
\end{itemize}

\subsubsection{Vércukor}
\begin{itemize}
    \item A tápcsatorna a táplálékkal felvett szénhidrátokat glükózra (szőlőcukor) bontja.
    \item A glükóz a bélfalon keresztül a vérbe kerül és ezúton a test minden részére eljut.
    \item Cukorhiánynál a májban egy folyamatos szőlőcukor-újraképzés (glukoneogenezis) zajlik és ez biztosítja a vércukor konstans szinten tartását.
    \item A sejtek a glükózt energiaforrásként használják valamint egyes sejtek (máj és izom) ezen kívül képesek a glükóz tárolására is szénhidrát (glikogén) formájában.
    \item Inzulin
    \begin{itemize}
        \item Latin insula (sziget) szóból kapta
        \item Langerhans német kutató (Langerhans szigetek (a hasnyálmirigy szöveti eleme) (1\%).)
        \item Ha egy állatból kivette a hasnyálmirigyet, akkor a cukorbetegség tünetei jelentek meg.
        \item Az inzulin serkenti a máj glikogénraktározását és a sejtek glükózfelvételét, ily módon csökkenti a vércukorszintet.
    \end{itemize}
    \item Ha az inzulin hiányzik (abszolút inzulinhiány) vagy nem tud rendesen hatni (relatív inzulinhiány), hiányzik az inzulin glukoneogenesist gátló hatása és a szabályozás felborul.
    \item A máj inzulin hiányában naponta 500 g szőlőcukort képes termelni.
    \item Az inzulin inzulinreceptorokon keresztül kötni tud a test egyes sejtjeihez (máj-, izom- és zsírsejtek) és kis pórusokat nyit a sejtmembránon, amin keresztül sejtek a glükózt fel tudják venni.
\end{itemize}

\subsection{Inzulin problémák}
\begin{itemize}
    \item Hiány
    \begin{itemize}
        \item A sejtek (az agysejtek kivételével) nem tudják a glükózt a vérből felvenni, így az a vérben marad.
        \item Növekedik a glükóz-újraképzés a májban.
        \item Megemelkedik a vércukorszint.
    \end{itemize}
    \item Felesleg
    \begin{itemize}
        \item Inzulinrezisztencia: a sejtek idővel ellenállnak az inzulin sejthártya nyitogató kísérleteinek.
    \end{itemize}
\end{itemize}

\subsection{Cukorbetegség}
\begin{itemize}
    \item Diabetes mellitus vagy diabétesz
    \item A cukor vizelettel való fokozott kiválasztására és a megemelkedett vizeletmennyiségre utal.
    \item A 2-es típusú cukorbetegség
    \begin{itemize}
        \item Lépcsőzetes kezelés
        \item Életmódváltoztatás $\rightarrow$ testsúlycsökkentés
        \item Tablettás antidiabetikus gyógyszerek
        \item A betegség előrehaladásával, amikor a béta-sejtek kimerülése megindul, a tablettás készítmények mellett szükség lehet hosszú hatású inzulinkészítmények esti adására.
        \item Az utolsó szakaszban, amikor a béta sejtek kimerültek, az inzulint az 1-es típusú diabéteszhez hasonlóan teljesen pótolni kell.        
    \end{itemize}
    \item 1-es típusú cukorbetegség
    \begin{itemize}
        \item Inzulin adás szükséges (rendszeres)
        \item A hasnyálmirigy inzulint termelő béta-sejtjeinek pusztulása következtében nincs elegendő inzulintermelés.
    \end{itemize}
    \item GDM - A terhességi vagy gesztációs diabetes mellitus
    \begin{itemize}
        \item A terhesség első három hónapjában jelentkezik és a terhesség végével általában eltűnik.
        \item A terhességi hormonok hatására alakulhat ki.
    \end{itemize}
\end{itemize}

\clearpage
\subsection{Vércukorszint}
\begin{itemize}
    \item A vérben lévő cukor mennyisége folyamatosan ingadozik (ez normális)
    \begin{itemize}
        \item Nem cukorbetegeknél $\rightarrow$ Étkezések előtt akár 4-6 mmol/1 között, étkezéseket követően pedig 5-9 mmol/1 között.
        \item Cukorbetegeknél az ingadozások mértéke ennél nagyobb lehet.
    \end{itemize}
    \item A rendszeres vércukormérés eredményeiből szabad csak következtetéseket levonni.
    \begin{itemize}
        \item Ha azok egy-egy napszakra, egy-egy étkezés előtt vagy után jellemzően kimutathatók, az egy-egy időpontban végzett mérések több, mint 70\%-ában reprodukálhatók.
    \end{itemize}
\end{itemize}

\subsection{Vércukormérés gyakorisága}
\begin{itemize}
    \item Diétával és tablettával kezelt cukorbetegeknél kisebbek a vércukoringadozások ezért általában elég naponta egy vércukorszint mérés. 
    \item Napi profilmérés $\rightarrow$ egyelten nap 5-6 vércukormérés, ezt követően pedig 1-2 héten át semmi.
    \item A lépcsőzetes – naponta, másnaponta egyszer – végzett vércukorszint mérések jól tükrözik a vércukor alakulás napszakos dinamikáját.
    \item Inzulinos pácienseknél naponta többször.
\end{itemize}

\subsection{Vércukormérés}
\begin{itemize}
    \item Vércukormérő berendezés
    \begin{itemize}
        \item Kisméretű, egyszerűen kezelhető leolvasó műszer
    \end{itemize}
    \item Tesztcsík
    \begin{itemize}
        \item Színelváltozások egyértelmű és minél pontosabb leolvasása
    \end{itemize}
    \item Kalibrációs segédeszközök egyszerű hitelesítés
    \item Lándra (vért veszünk szúrással)
\end{itemize}

\subsection{Hátrányok}
\begin{itemize}
    \item A kémia reakcióknál a vegyi anyagok szavatossága korlátos
    \item Leolvasás automatizáltsága
    \item Eredmények reprodukálhatósága, validálás
\end{itemize}

\subsection{CGM - Folyamatos vércukormonitorozás napjainkban}
\begin{itemize}
    \item Abbot Freestyle Libre
    \item DEXCOM
    \begin{itemize}
        \item DEXCOM G5
        \item DEXCOM G6
    \end{itemize}
    \item MEDTRONIC - Guardian
\end{itemize}

\subsection{Inzulin}
\begin{itemize}
    \item Régebben állati inzulin
    \begin{itemize}
        \item Sertés és marha hasnyálmirigyéből kivont
        inzulinkészítményeket ma kizárólag azok
        kapnak, akik már régóta ezeket a
        készítményeket használják. (esetenként súlyos
        allergiás reakciók)
        \item E.-coli nevű baktériummal, sütőélesztővel
        állítják elő (inzulin termelésére
        beprogramozva) a "humán inzulint"
    \end{itemize}
\end{itemize}

\subsubsection{Inzulin beadása}
\begin{itemize}
    \item Régen injekciós tűvel (inzulin)
    \item Inzulin toll segítségével
    \item Inzulin pumpa segítségével
    \item Kísérleti fázis:
    \begin{itemize}
        \item Orron át adható $\rightarrow$ orrspray
        \item Szájon át adható $\rightarrow$ folyadék, tabletta
        \item Belégzéssel $\rightarrow$ inhalációs inzulin
    \end{itemize}
\end{itemize}

\subsubsection{Inzulin adagolók napjainkban}
\customwidthimage{inz}{15cm}

\subsubsection{Auto-injector megoldás belső részei}
\customwidthimage{autinj}{10cm}

\subsubsection{CGM + inzulinpumpa}
\begin{itemize}
    \item Folyamatos vércukorszint mérés kombinálva inzulinpumpával
    \item Mesterséges hasnyálmirigy
    \item Automatikus szabályozó algoritmusok
    \item Inzulin, és/vagy szénhidrát adagolás
\end{itemize}

\clearpage
\subsubsection{Koleszterin (LDL/HDL), laktát, hemoglobin,triglicerid mérés}
\begin{itemize}
    \item Mérő berendezés
    \begin{itemize}
        \item Kisméretű, egyszerűen kezelhető leolvasó műszer
    \end{itemize}
    \item Tesztcsík (többnyire külön az egyes mérési paraméterekre)
    \begin{itemize}
        \item Színelváltozások egyértelmű és minél pontosabb leolvasása
    \end{itemize}
    \item Kalibrációs segédeszközök egyszerű hitelesítés
    \item Lándzsa (vért veszünk szúrással)
    \item Digitális
\end{itemize}

\subsection{Gyógyszerszedés}
\begin{itemize}
    \item Széles a piac
    \item Idősebbek többet szednek
    \item Gyógyszer adagoló megoldások
    \begin{itemize}
        \item Elektromos/mechanikus
        \item Gyógyszer adagolók
        \item Elektromos, automata porciózással
        \item Manuális-mechanikus felhasználói aktivitással
    \end{itemize}
\end{itemize}

\subsection{Evolúció}
\customwidthimage{ev}{15cm}

\subsection{Defibrillátorok és pacemaker-ek}
\begin{itemize}
    \item Pacemaker
    \item Defibrillator (ICD)
    \item Cardiac Resynchronization Therapy (CRT)
\end{itemize}

\subsection{Az emberi légzés monitorozása}
\begin{itemize}
    \item Külső légzés
    \begin{itemize}
        \item Állandó légcsere zajlik a tüdő és a környezet között
    \end{itemize}
    \item Belső légzés
    \begin{itemize}
        \item A sejtek és szövetek légzése (gázcsere)
    \end{itemize}
\end{itemize}

\subsubsection{Pulzoximetria}
\begin{itemize}
    \item Mérő szenzor a páciens ujjára, ami a mérés során folyamatosan érzékeli az oxigén szint változását.
    \item Kis csipeszhez hasonló készülék
    \item Fájdalommentes vizsgálat
    \item Kritikus állapotú betegben a módszer sajnos gyakran megbízhatatlan a perifériás vasoconstrictio miatt (az eszköz nem érzékeli a pulzushullámot).
\end{itemize}

\subsubsection{Külső légzést monitorozó eszközök}
\begin{itemize}
    \item Kilégzési csúcsáramlás mérő
    \begin{itemize}
        \item Lényegében azt méri, hogy egy kilégzés alkalmával milyen gyorsan jut ki a levegő a tüdőből.
        \item A leolvasott érték az úgynevezett kilégzési csúcsáramlás értéke (peak expiratory flow, PEF), egysége l/perc.        
        \item Standard értéktáblázat, amiben benne vannak a testmagasság, életkor és nem szerint specifikált értékek.
    \end{itemize}
    \item Spirométer
    \begin{itemize}
        \item A tüdő levegőbefogadó képességének mérésére való eszköz
    \end{itemize}
    \item Légzési rátát monitorozó öv
    \begin{itemize}
        \item Nyúlásképes pánt, ami a mellkas térfogatváltozását érzékeli
    \end{itemize}
    \item Légzésfigyelő/őr babákhoz
    \begin{itemize}
        \item Babaágy aljára helyezhető mozgás érzékelő betét/matrac
    \end{itemize}
    \item Légzésszámláló
    \begin{itemize}
        \item Alvás/horkolás monitorozás mikrofonnal
    \end{itemize}
\end{itemize}

\clearpage
\subsection{Mozgásmonitorozás vs. mozdulatmonitorozás}
\begin{itemize}
    \item Mozgásfigyelés
    \begin{itemize}
        \item Mozgás a lakásban
        \item Életjel (anno: füst felszáll a szomszéd kéményéből)
        \item Alvásmonitorozás
    \end{itemize}
    \item Mozdulatmonitorozás
    \begin{itemize}
        \item Rehabilitáció (stroke, baleset, fejlődési rendellenesség)
        \begin{itemize}
            \item Helyesen/helytelenül végzett gyakorlat (szög, sebesség, táv, mennyiség)
        \end{itemize}
        \item Biomechanikai elemzések
        \begin{itemize}
            \item Hogyan csinálja, mozdulat optimalizálás
        \end{itemize}
        \item Mozgástanulás
    \end{itemize}
\end{itemize}

\subsection{Mozdulat és mozgásmonitorozó rendszerek}
\begin{itemize}
    \item Mozgásmennyiség szenzorok
    \begin{itemize}
        \item Testen viselt szenzorok $\rightarrow$ Aktigráfok, aktivitás érzékelők
        \item Lakáson belüli szenzorok $\rightarrow$ Passzív/aktív falra szerelhető mozgás szenzorok
    \end{itemize}
    \item Monitorozás optikai tartományban
    \begin{itemize}
        \item Kamera alapú rendszerek passzív optikai referencia markerekkel
    \end{itemize}
    \item Monitorozás rádió és/vagy ultrahang tartományban
    \begin{itemize}
        \item Aktív, vagy passzív markerekkel
    \end{itemize}
\end{itemize}

\clearpage
\subsection{Személyi mozgásmennyiség mérő}
\begin{itemize}
    \item Mozgásérzékelő szenzor (accelerométer), ami alkalmas a különböző végtag és törzsmozgások detektálására $\rightarrow$ Gyorsulásmérő
    \item Részei
    \begin{itemize}
        \item A piezo-elektromos gyorsulásmérő/giroszkóp, stb
        \item Esemény szűrő
        \item Belső óra és memória
    \end{itemize}
\end{itemize}

\subsection{Mozgásmennyiség figyelés (életviteli minták)}
\begin{itemize}
    \item Passzív mozgásérzékelők
    \item Telepített, viszonylag állandó/stabil rendszerek
\end{itemize}

\subsection{Mozdulat monitorozás}
\begin{itemize}
    \item Cél $\rightarrow$ A mozdulat/testmozgás kvalitatív és kvantitatív jellemzőinek mérése és rögzítése
    \begin{itemize}
        \item Mozdulat sebessége
        \item Mozdulat pontossága
        \item Mozdulatok száma
        \item Izületi mozgásterjedelem
    \end{itemize}
\end{itemize}

\clearpage
\subsection{A monitorozás eredményeinek felhasználási területei}
\begin{itemize}
    \item Mozgásrehabilitáció
    \begin{itemize}
        \item Balesetek utókezelése
        \item Stroke utókezelése
    \end{itemize}
    \item Mozgásszervi betegségek kezelése
    \item Tremor (nyugalmi, akciós, stb) regisztrálása és vizsgálata (Parkinson)
    \item Távrehabilitáció/gyógytorna
    \item Sportmozgások biomechanikai elemzése
    \item Mozdulat/mozgástanulás támogatása
    \item Motion capture (pl.: filmek)
\end{itemize}

\subsection{Megoldási alternatívák}
\begin{itemize}
    \item Hagyományos módszer
    \item Video/markeres módszerek
    \item Célhardverek
    \item Informatikai megoldások $\rightarrow$ Alternatív/költséghatékony
\end{itemize}

\subsection{Mozgásmonitorozás - célhardverek}
\begin{itemize}
    \item Hang RF/Ultrahang
    \item Speciális ruha (aktív-passzív markerek)
    \item IR grid
\end{itemize}

\subsection{Eszközök}
\begin{itemize}
    \item Hardver
    \begin{itemize}
        \item Zebris (német)
        \item CMAS (amerikai)
        \item Cricker Indoor Location System (amerikai)
        \item Kinect I/II (Xbox, Windows)
        \item IMU alapokon
    \end{itemize}
    \item Szoftver
    \begin{itemize}
        \item SkillSpector (videó alapú)
        \item The MotionMonitor (videó vagy hardver alapú)
    \end{itemize}
\end{itemize}

\subsection{Mozdulatkövetés a gyakorlatban (hardver és eszköz)}
\begin{itemize}
    \item Hardver \#1
    \begin{itemize}
        \item Hibrid megoldás (RF + ultrahang)
        \item RF frekvencia (433 MHz)
        \item 30 méteres hatótávolság
        \item Felbontóképesség (1cm (3 méteren), 2 cm (10 méteren))
    \end{itemize}
\end{itemize}

\subsection{Szoftver képességek}
\begin{itemize}
    \item Mozgási adatok beolvasása (3D térkoordináták, idő és szenzor adatok)
    \begin{itemize}
        \item Külső mozgásmonitorozó eszközről + archivált adatokból
    \end{itemize}
    \item Mozgás grafikus megjelenítése
    \item Számítások (távolság és hajlásszög számítás)
    \item Mozgási adatok rögzítése, minta mozgások felvétele
    \item Korábban rögzített mozgási adatok elemzése
\end{itemize}