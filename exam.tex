\section{Vizsga kérdések}

\subsection{Szenzorok}
\subsubsection{Szenzor definíció}
\begin{itemize}
    \item A szenzor egy eszköz, ami fizikai ingereket (pl.: hőmérséklet, fény, nyomás) érzékel és ezeket mérhető jelekké alakítja.
    \item Ezek a jelek lehetővé teszik környezetünk változásainak észlelését és mérését. (pl.: ipar, egészségügy, környezetvédelem)
    \item Szenzorok lehetnek aktív vagy passzív típusúak, azaz saját energiát használó vagy a környezeti energiát felhasználó eszközök.
    \item Digitális adatot ad ki magából.
    \item A digitális adatot tovább adja küldeni vezetékes vagy vezeték nélküli csatornán.
\end{itemize}

\subsubsection{Szenzor kategóriák}
\begin{itemize}
    \item Hőmérséklet-szenzorok
    \begin{itemize}
        \item Környezet hőmérsékletének mérése, gyakran használják otthoni fűtési rendszerekben, autókban és ipari folyamatokban.
    \end{itemize}
    \item Gyorsulásmérők
    \begin{itemize}
        \item Gyorsulás/rezgés mértékét érzékelik, gyakran használják mobiltelefonok képernyő-orientációjának szabályozásához, járművek ütközésérzékelésére, épületek szeizmikus monitorozására.
    \end{itemize}
    \item Kémiai szenzorok
    \begin{itemize}
        \item Vegyületek mérése levegőben, vízben vagy egyéb közegben.
        \item Környezetszennyezés monitorozása, ipari folyamatok ellenőrzése, egészségügyben.
    \end{itemize}
    \item Hangfrekvencia-szenzorok
    \begin{itemize}
        \item Hanghullámokat érzékel, alkalmazhatóak biztonsági rendszerekben, hangfelismerésben és akusztikai elemzésben.
    \end{itemize}
\end{itemize}

\subsubsection{Szenzorok használati területei}
\begin{quote}
    A szenzorok javítják életünk minőségét, hatékonyságát és biztonságát.
\end{quote}
\begin{itemize}
    \item Okostelefonok és viselhető eszközök
    \begin{itemize}
        \item Gyorsulásmérők, giroszkópok, lépésszámlálás, automatikus fényerő-szabályozás.
    \end{itemize}
    \item Otthoni automatizálás és intelligens otthonok
    \begin{itemize}
        \item Hőmérséklet, fény, mozgás
    \end{itemize}
    \item Ipari automatizálás
    \begin{itemize}
        \item Üzemeltetés hatékonyságát növeli.
    \end{itemize}
    \item Egészségügy és orvostechnika
    \begin{itemize}
        \item Kémiai szenzorok, biometrikus szenzorok, diagnosztikák/kezelések monitorozása
    \end{itemize}
    \item Autóipar
    \begin{itemize}
        \item Ütközésérzékelők, parkolássegítő rendszerek
    \end{itemize}
    \item Biztonsági rendszerek
    \begin{itemize}
        \item Mozgásérzékelők, füstérzékelők, kamera szenzorok a veszélyhelyzetek azonosítására.
    \end{itemize}
\end{itemize}

\clearpage
\subsubsection{Szenzor problémák}
\begin{itemize}
    \item Kalibrációs problémák
    \begin{itemize}
        \item Következménye a pontatlanság lehet, kell a rendszeres kalibráció a pontos működéshez.
    \end{itemize}
    \item Környezeti hatások
    \begin{itemize}
        \item Extrém hőmérsékletek, por és egyéb környezeti tényezők befolyásolhatják a szenzorok teljesítményét.
    \end{itemize}
    \item Energiafogyasztás
    \begin{itemize}
        \item Akkumulátorral működő eszközöknél a szenzorok sokat fogyasztanak.
    \end{itemize}
    \item Inferencia és zaj
    \begin{itemize}
        \item Zajok torzíthatják a szenzorok által gyűjtött adatokat, ami pontatlansághoz vezet.
    \end{itemize}
    \item Technológiai korlátok (Például kommunikáció)
\end{itemize}

\subsubsection{Aktuátor definíció, példák}
\begin{quote}
    Az aktuátorok olyan eszközök, amik elektromos jelet fizikai műveletekké alakítanak át, az aktuátorok lehetnek mechanikus szerkezetek vagy bonyolultabb rendszerek.
\end{quote}
\begin{itemize}
    \item Elektromos motorok
    \begin{itemize}
        \item Elektromos energiát mechanikai mozgássá alakítanak át, járművek meghajtását teszik lehetővé például.
    \end{itemize}
    \item Hidraulikus aktuátorok
    \begin{itemize}
        \item Folyadék nyomásának növelésével/csökkentésével működnek, erős és precíz mozgatásra képesek, például építőipari gépekben.
    \end{itemize}
    \item Pneumatikus aktuátorok
    \begin{itemize}
        \item Sűrített levegőt használnak a mozgás előidézésére, például ahol gyors és ismétlődő mozgásra van szükség.
    \end{itemize}
\end{itemize}

\subsection{DAQ/Tradícionális vezérlés/DCS/SCADA/Monitoring/Vezérlőrendszerek}
\subsubsection{DIKW piramis}
\begin{quote}
    Tudáspiramis, egy modell, bemutatja hogyan alakulnak át az adatok értelmezhető és használható tudássá.
\end{quote}
\begin{enumerate}
    \item Adat (Nyers adatok kontextus nélkül (pl.: mérések adatai), kevés hasznos információt tartalmaznak.)
    \item Információ (Adatok kontextusba helyezése.)
    \item Tudás (Információkból következtetéseket lehet levonni.)
    \item Bölcsesség (Tudás alkalmazása, miértje.)
\end{enumerate}

\subsubsection{Tradícionális vezérlési rendszerek vs. Elosztott vezérlési rendszerek (DCS)}
\begin{itemize}
    \item Tradícionális vezérlési rendszerek
    \begin{itemize}
        \item Központosított architektúrára épül, ahol egy vagy több központi vezérlőegység végzi a folyamatok összes vezérlési és felügyeleti feladatát.
        \item Egyszerű, alacsony költségek
        \item Korlátozott skálázhatóság, a központosított vezérlés miatt nagyobb a rendszer kiesésének kockázata
    \end{itemize}
    \item Elosztott vezérlési rendszerek (DCS)
    \begin{itemize}
        \item Vezérlési folyamatokat moduláris egységek között osztja szét, amik kommunikálnak egymással egy közös hálózaton keresztül.
        \item Minden egyes szegmens a saját területéért felelős.
        \item Magasabb rendelkezésre állás, megbízhatóság, skálázhatóság
        \item Költséges, bonyolult
    \end{itemize}
\end{itemize}

\subsubsection{DAQ rendszer előnyök/hátrányok, alkalmazási területek}
\begin{quote}
    Adatgyűjtő rendszerek, lehetővé teszik fizikai jelenségek valós idejű monitorozását és analízisét. \\
    Egy tipikus DAQ rendszer szenzorokból áll, adatgyűjtő hardverből és szoftverből áll, amik összegyűjtik és feldolgozzák az adatokat.
\end{quote}
\begin{itemize}
    \item Előnyök (Rugalmas konfiguráció, pontos mérések, valós idejű adatfeldolgozás, automatizálás)
    \item Hátrányok (Költségek, technikai komplexitás, hardver és szoftverkompatibilitás, Karbantartás, frissítések)
    \item Alkalmazási területek (Tudományos kutatás, mérnöki tesztelés és fejlesztés, környezeti monitorozás, egészségügy)
\end{itemize}

\subsubsection{Elosztott vezérlési rendszerek (DCS)}
\begin{itemize}
    \item Előnyök (Magasabb rendelkezésre állás, megbízhatóság, skálázhatóság)
    \item Hátrányok (Költséges, bonyolult)
    \item Alkalmazási területek (Gyógyszeripar, erőművek, élelmiszeripar)
\end{itemize}

\subsubsection{SCADA rendszer}
\begin{itemize}
    \item Előnyök (Távoli felügyelet, valós idejű adatgyűjtés, megbízhatóság, automatizálás)
    \item Hátrányok (Komplexitás, költségek, karbantartás)
    \item Alkalmazási területek (Energiaipar, közlekedés, gyártás és automatizálás)
\end{itemize}

\clearpage
\subsection{DAQ}
\subsubsection{DAQ rendszer komponensei}
\begin{itemize}
    \item Szenzorok és érzékelők (Fizikai változások elektromos jelekké alakítja át)
    \item Jelkondicionáló áramkörök (Jeleket átalakítja, hogy azok megfelelőek legyenek)
    \item Adatgyűjtő eszközök - DAQ hardver (Előkészített analóg jeleket digitális formátumba konvertálja)
    \item Számítógép és interfész (DAQ hardvert számítógéphez kell csatlakoztatni, interfész lehet USB, PCI, PCIe, Ethernet és szoftveren keresztül kezeli az adatokat)
    \item Szoftver (LabVIEW, MATLAB)
\end{itemize}

\subsubsection{DAQ alkalmazási területek}
\begin{itemize}
    \item Tudományos kutatás, mérnöki tesztelés és fejlesztés, környezeti monitorozás, egészségügy
\end{itemize}

\subsubsection{Jelkondícionálás}
\begin{quote}
    A jelkondícionálás előkészíti az elektromos jeleket a digitális átalakításra és feldolgozásra, célja, hogy javítsa a jelek minőségét és növelje az adatgyűjtés pontosságát.
\end{quote}
\begin{itemize}
    \item Erősítés, szűrés, hőmérséklet-kompenzáció, lineárizáció, galvanikus leválasztás, jelalakítás
\end{itemize}

\subsubsection{Betegmonitorozó DAQ infrastruktúra}
\begin{quote}
    Beteg valós időben történő monitorozása.
\end{quote}
\begin{itemize}
    \item Szenzorok és érzékelők, jelkondícionáló áramkörök, adatgyűjtő eszközök, központi monitorozó állomás és hálózati infrastruktúra, szoftver és analitikai eszközök, adattárolás és archiválás
\end{itemize}

\subsubsection{DAQ rendszerek kihívásai}
\quote
\begin{itemize}
    \item Jelzaj és interferencia, nagy adatmennyiség kezelése, szenzor kalibráció és hőmérsékleti hatások, Adatbiztonság és adatvédelem, hardver és szoftverkompatibilitás, skálázhatóság és rugalmasság, Kezelhetőség 
\end{itemize}

\subsection{Elosztott vezérlési rendszerek (DCS)}
\subsubsection{DCS architektúra}
\begin{itemize}
    \item Vezérlési folyamatokat moduláris egységek között osztja szét, amik kommunikálnak egymással egy közös hálózaton keresztül.
    \item Minden egyes szegmens a saját területéért felelős.
    \item Moduláris
\end{itemize}

\subsubsection{DCS komponensei}
\begin{itemize}
    \item Folyamatvezérlők
    \item Operátori állomások (HMI)
    \item I/O modulok
    \item Kommunikációs hálózatok
    \item Mérnöki munkaállomások
    \item Adatarchiváló és elemző rendszer
    \item Biztonsági rendszerek
\end{itemize}

\subsubsection{DCS előnyök és hátrányok}
\begin{itemize}
    \item Előnyök (Magasabb rendelkezésre állás, megbízhatóság, skálázhatóság)
    \item Hátrányok (Költséges, bonyolult)
\end{itemize}

\subsubsection{DCS alkalmazási területek}
\begin{itemize}
    \item Gyógyszeripar, erőművek, élelmiszeripar
\end{itemize}

\subsection{Supervisory control and data acquisition - SCADA}
\subsubsection{SCADA architektúra, komponensei}
\begin{quote}
    Lehetővé teszik a nagy ipari és infrastrukturális folyamatok távoli monitorozását, vezérlését és automatizálását.
\end{quote}
\begin{itemize}
    \item Terepi eszközök
    \item Távközlési rendszerek
    \item RTU-k
    \item SCADA szerverek és számítógépek
    \item HMI
    \item Adatbázis és archiváló rendszerek
    \item Biztonsági komponensek
    \item Alkalmazási és szoftvereszközök
\end{itemize}

\subsubsection{SCADA előnyök/hátrányok}
\begin{itemize}
    \item Előnyök (Távoli felügyelet, valós idejű adatgyűjtés, megbízhatóság, automatizálás)
    \item Hátrányok (Komplexitás, költségek, karbantartás)
\end{itemize}

\subsubsection{SCADA alkalmazási területek}
\begin{itemize}
    \item Energiaipar, közlekedés, gyártás és automatizálás
\end{itemize}

\clearpage
\subsubsection{SCADA funkciói}
\begin{itemize}
    \item Távmérések, távjelzések fogadása
    \item Visszajelzés, adat vizualizáció
    \item Naplózás
    \item Riasztások (határérték és gradiens figyelés)
    \item Topológia analízis
    \item Távparancsadás
    \item Autentikáció és jogosultságkezelés
    \item Adattárolás
\end{itemize}

\subsubsection{Basic SCADA vs. Integrated SCADA vs. Networked SCADA}
\begin{itemize}
    \item Basic SCADA
    \begin{itemize}
        \item Alapvető távoli felügyelet és adatgyűjtés
        \item Korlátozott I/O kapacitás, egyszerű HMI és adatgyűjtés
        \item Kis teljesítményűek, egyszerű ipari folyamatok
    \end{itemize}
    \item Integrated SCADA
    \begin{itemize}
        \item Bonyolultabb, több funkciót integrálnak egyetlen koherens rendszerben.
        \item ERP rendszerek, komplex gyártási folyamatok, nagy létesítmények, vállalati szintű
    \end{itemize}
    \item Networked SCADA
    \begin{itemize}
        \item Széleskörű hálózati kapcsolatok, távoli elérés és vezérlés, adatmegosztás a létesítmények között, felhőalapú adattárolás és szolgáltatások
        \item Távvezetéki rendszerek, vízellátás, energiaelosztás, szétszórt infrastruktúra felügyelet
    \end{itemize}
\end{itemize}

\clearpage
\subsection{Biojel-gyűjtés (Biosignal acquisition)}
\begin{quote}
    A bioszignálok az élőlények testéből származó elektromos, mechanikai vagy más fizikai jelek, amik információt hordoznak az adott szervezet vagy szervrendszer állapotáról.
\end{quote}

\begin{itemize}
    \item Elektromos bioszignálok (Ideg és izomsejtek elektromos aktivitásából származnak)
    \begin{itemize}
        \item \textbf{EKG} (Elektrokardiogram), szív elektromos tevékenysége.
        \item \textbf{EEG} (Elektroenkefalográfia), Agy elektromos aktivitása.
        \item \textbf{EMG} (Elektromiográfia), izom elektromos aktivitása.
    \end{itemize}
    \item Mechanikai bioszignálok (Fizikai mozgások és változások)
    \begin{itemize}
        \item \textbf{Pulzushullám-velocitás}, az artériás rugalmasság mérésére szolgáló jel.
        \item \textbf{Spirometria}, légzés mechanikájának mérése.
    \end{itemize}
    \item Kémiai és biokémiai bioszignálok (Kémiai összetétel változásai)
    \begin{itemize}
        \item \textbf{Glükózszint-mérés}, vércukorszint mérés.
        \item \textbf{pH-mérés}, testfolyadék savasságának mérése.
    \end{itemize}
    \item Optikai és termikus bioszignálok (Fény és hő alapú jelek)
    \begin{itemize}
        \item \textbf{Pulzoximetria}, véroxigénszint mérés.
        \item \textbf{Testhőmérséklet-mérés}, a test belső hőmérsékletének mérési módjai.
    \end{itemize}
    \item Környezeti tényezők és egyéb mérések
    \begin{itemize}
        \item \textbf{Páratartalom és hőmérséklet}
        \item \textbf{Tartás és gyorsulás}
    \end{itemize}
\end{itemize}

\subsubsection{Szenzoradat kezelési folyamat}
\begin{enumerate}
    \item Data acquisition (DAQ) (egy/több szenzor)
    \item Adatkezelés (feldolgozás/szűrés)
    \item Tárolás, keresés
    \item Vizualizáció
    \item Megosztás
\end{enumerate}

\subsubsection{Mérési hibák (típusok/források)}
\begin{quote}
    Minden mérés tartalmaz hibákat!
\end{quote}
\begin{itemize}
    \item Rendszerszerű hibák
    \begin{itemize}
        \item \textbf{Kalibrációs hibák} (Mérőeszközök nem megfelelő kalibrálása)
        \item \textbf{Elektromos interferencia} (Környezeti elektromos berendezések zavarai)
        \item \textbf{Jelátviteli hiba} (Hosszú vagy rossz minőségű kábelezés)
    \end{itemize}
    \item Random hibák
    \begin{itemize}
        \item \textbf{Operátori hiba} (Emberi tényező, mint pl.: szenzor helytelen elhelyezése)
        \item \textbf{Fiziológiai zaj} (A testből származó nem kívánt jelek, pl.: izomzaj)
        \item \textbf{Mintavételi hiba} (Nem megfelelő mintavételi frekvencia alkalmazása)
    \end{itemize}
\end{itemize}

\clearpage
\subsubsection{Egyetlen vs. több szenzoros mérési problémák}
\begin{itemize}
    \item Egyetlen szenzoros mérési problémák
    \begin{itemize}
        \item \textbf{Korlátozott információszerzés}, egyetlen nézőpontból származó adatok korlátozott betekintést nyújtanak.
        \item \textbf{Hibatűrés hiánya}, egy szenzor meghibásodásánál nincs redundancia, ami az egész mérési folyamat kieséséhez vezethet.
        \item \textbf{Nagyobb kockázat a pontatlanságokra}, az adatok értelmezésekor
    \end{itemize}
    \item Több szenzoros mérési problémák
    \begin{itemize}
        \item \textbf{Adatkezelés és feldolgozás}, nagy adatmennyiség kezelése, tárolása, elemzése bonyolult, időigényes
        \item \textbf{Adatfúzió és integráció}, a különböző típusú és forrású adatok összeegyeztetése és integrálása technikai kihívást jelenthet.
        \item \textbf{Interferencia és koherencia}, a szenzorok közötti interferencia és az adatok koherenciájának hiánya torzíthatja az eredményeket.
    \end{itemize}
\end{itemize}

\clearpage
\subsubsection{Mérnöki kihívások a nagymennyiségű adatgyűjtésnél}
\begin{itemize}
    \item Általános kihívások
    \begin{itemize}
        \item P1 $\rightarrow$ Sok DAQ csomópont
        \item P2 $\rightarrow$ Sok szenzor (különböző típusú)
        \item P3 $\rightarrow$ Nagy adatmennyiség lehetőleg gyorsan átküldve
    \end{itemize}
    \item Kommunikációs probléma $\rightarrow$ P1 $\times$ P2 $\times$ P3
    \item Szoftver kihívások
    \begin{itemize}
        \item Valós idejű DAQ + előfeldolgozás + feldolgozás + megjelenítés (különböző tartományok)
        \item Komplex döntési helyzetek
        \item Online adatmenedzsment (megosztás + archiválás)
    \end{itemize}
    \item Hardver korlátok
    \begin{itemize}
        \item Energia problémák
        \item Kommunikációs hatótávolságok, adat multiplexálási problémák/idő, frekvencia
        \item Biztonság, megbízhatóság, használhatóság
    \end{itemize}
\end{itemize}