\section{Téma 1}

\subsection{Szenzor/érzékelő}
\begin{itemize}
    \item Monitorozásnál a szenzor
    \begin{itemize}
        \item Érzékelő ami valamilyen fizikai/kémiai mennyiséget, vagy annak változását méri.
        \item Digitális adatot ad ki magából.
        \item A digitális adatot tovább tudja küldeni \textbf{vezetékes}, vagy \textbf{vezeték nélküli} csatornán.
    \end{itemize}
    \item Szenzor problémák
    \begin{itemize}
        \item Elem
        \item Kezelhetőség
        \item Működési időtartam
        \item Élettartam
        \item Kábelek, környezeti beépítési problémák
        \item Egészségkárosító hatás
        \item Technológiai korlátok (kommunikációs hatótávolság)
    \end{itemize}
\end{itemize}

\subsection{Adat-Információ-Ismeret-Tudás piramis}
\begin{center}
    \begin{tikzpicture}[every node/.style={draw, anchor=south, inner sep=5pt}]
        % Define nodes
        \node[trapezium, trapezium angle=60, minimum width=5cm, fill=red!30] (data) {Adat};
        \node[trapezium, trapezium angle=60, minimum width=4cm, fill=orange!40, above=0.5cm of data] (info) {Információ};
        \node[trapezium, trapezium angle=60, minimum width=3cm, fill=yellow!60, above=0.5cm of info] (knowledge) {Ismeret};
        \node[trapezium, trapezium angle=60, minimum width=2cm, fill=green!70, above=0.5cm of knowledge] (wisdom) {Tudás};
    
        % Define labels on the right side
        \node[anchor=west, draw=none, fill=none] (dataLabel) at ([xshift=2cm]data.east) {Inputok};
        \node[anchor=west, draw=none, fill=none] (infoLabel) at ([xshift=2cm]info.east) {Belső infó};
        \node[anchor=west, draw=none, fill=none] (knowledgeLabel) at ([xshift=2cm]knowledge.east) {Akciók};
        \node[anchor=west, draw=none, fill=none] (wisdomLabel) at ([xshift=2cm]wisdom.east) {Eredmények};
    
        % Draw arrows with some space between the trapeziums and the arrows on the right
        \draw[->] (dataLabel.west) -- ([xshift=0.4cm]data.east);
        \draw[<-] (infoLabel.west) -- ([xshift=0.4cm]info.east);
        \draw[<-] (knowledgeLabel.west) -- ([xshift=0.4cm]knowledge.east);
        \draw[<-] (wisdomLabel.west) -- ([xshift=0.4cm]wisdom.east);
    
        % Define labels on the left side
        \node[anchor=east, draw=none, fill=none] at ([xshift=-3cm]info.west) (dataValue) {+ érték};
        \node[anchor=east, draw=none, fill=none] at ([xshift=-3.5cm]knowledge.west) (infoValue) {+ érték};
    
        % Place "Feldolgozás" at the top arrow
        \node[anchor=east, draw=none, fill=none] at ([xshift=-2.5cm]wisdom.west) (process) {Feldolgozás};
    
        % Draw arrows vertically between the levels on the left side
        \draw[->] ([xshift=-3.8cm]data.south) -- ([xshift=-2.4cm]info.north);
        \draw[->] ([xshift=-3.8cm]info.south) -- ([xshift=-2.4cm]knowledge.north);
        \draw[->] ([xshift=-3.8cm]knowledge.south) -- ([xshift=0.5cm]process.east);
    
    \end{tikzpicture}
\end{center}

\subsection{Data Acquisition - DAQ rendszerek}
\begin{itemize}
    \item Szenzorok
    \item Kliens oldali adatgyűjtő eszközök
    \begin{itemize}
        \item Megjelenítési felülettel, vagy anélkül
        \item Általános célú megoldások (PC, mini PC, okos telefon)
        \item Célhardverek
    \end{itemize}
    \item Központi adat tároló és adat menedzselő rendszerek (adat szerverek)
    \item Központi/kliens oldali megjelenítő rendszerek (web/alkalmazás szerverek)
    \item Központi rendszer felügyeleti eszközök (alkalmazás szerverek)
    \item Üzemeltető és kiszolgáló személyzet infrastruktúrája (szakemberek, technikusok, CallCenter, stb.)
    \item Kommunikációs infrastruktúra, ami az egyes részeket összeköti.
\end{itemize}

\subsection{Járulékos hardver eszközök}
\begin{itemize}
    \item Összeköttetéshez a központi irányába
    \begin{itemize}
        \item Okostelefon
        \item Számítógép/célszámítógép/PDA
    \end{itemize}
    \item Kábelek a szenzor és az adatgyűjtő összeköttetéshez
    \item Dongle-ek a vezetéknélküli átvitelhez
\end{itemize}

\subsection{Rendszerirányító rendszerek}
\begin{itemize}
    \item Valós idejű információ folyamatos továbbítása, tárolása és feldolgozása, a rendszerirányítás megbízható számítógépes támogatása mind az operatív üzemirányítás, mind az üzemelőkészítés és üzemértékelés elvégzéséhez.
    \item Központi vezérlési rendszerek
    \item Distributed Control System - DCS
    \item Supervisory control and data acquisition - SCADA
\end{itemize}

\subsubsection{Distributed Control System - DCS}
\begin{itemize}
    \item Szemben a direkt helyi vezérlővel megvalósított rendszerekkel:
    \begin{itemize}
        \item a DCS-ek esetében nagyobb a megbízhatóság,
        \item kisebbek a kialakítási költségek, mivel a vezérlés lokálisan megvalósított és a vezérléshez szükséges kommunikáció is lokális,
        \item központi (akár távolról megvalósított) ellenőrzés
    \end{itemize}
\end{itemize}

\subsubsection{Supervisory Control and Data Acquisition - SCADA}
\begin{itemize}
    \item Felügyeleti szabályozás és adatgyűjtés
    \begin{itemize}
        \item Központosított vezérlés (figyel és vezérel)
        \item Távoli adatgyűjtől
        \item Biztonságos kommunikáció
        \item Elosztott adattárolás
        \item Időbélyegek
        \item Például $\rightarrow$ Erőművek, csővezetékek, elektromos hálózatok, vízhálózatok
    \end{itemize}
    \item \underline{\textbf{S}}upervisory $\rightarrow$ Operator, engineer, supervisor
    \item \underline{\textbf{C}}ontrol $\rightarrow$ Monitoring, Limited, Telemetry, Remote/Local
    \item \underline{\textbf{D}}ata \underline{\textbf{A}}cquisition $\rightarrow$ Analog/Digital
\end{itemize}

\customwidthimage{scada}{8cm}

\subsubsection{SCADA rendszer generációk}
\begin{itemize}
    \item Monolitikus, szigetszerű rendszerek, egymástól függetlenül működő zárt (kapcsolat nélküli)
    \item Elosztott rendszerek $\rightarrow$ LAN hálózatba rendezett, egymással hálózati protokollokon kommunikáló
    \item Hálózatos rendszerek $\rightarrow$ Földrajzilag nagyobb kiterjedésű (LAN-nál nagyobb) hálózatokba rendezett, egymással hálózati protokollokon kommunikáló rendszerek, egymástól független, egymással párhuzamosan futó
    \item Internet of Things (IoT)
    \item Felhő infrastruktúrákkal támogatott, növelt hatékonysággal, és optimált költségekkel üzemelő online, valós idejű rendszerek.
\end{itemize}

\subsubsection{SCADA rendszer általános belső architektúrája}
\customwidthimage{scada_arch}{15cm}

\clearpage
\subsubsection{A SCADA rendszerek főbb funkciói}
\begin{itemize}
    \item Távmérések, távjelzések fogadása
    \item Visszajelzés, adat vizualizáció
    \item Naplózás
    \item Riasztások (határérték és gradiens figyelés)
    \item Topológia analízis
    \item Távparancsadás
    \item Autentikáció és jogosultságkezelés
    \item Adattárolás
\end{itemize}

\subsection{Példák a távoli adatgyűjtésre}
\begin{quote}
    Régebben távoli, nehezen megközelíthető, veszélyes helyeken mérés, illetve gyakori automatizált méréssorozatok esetén elterjedt.
\end{quote}
\begin{itemize}
    \item Mérő állomásoknál
    \begin{itemize}
        \item Meterológiai, vízállás mérő állomások, épületek/hidak/alagutak
    \end{itemize}
    \item Komplex berendezések
    \begin{itemize}
        \item Részecskegyorsítók
    \end{itemize}
    \item Egészséges emberek monitorozása
    \begin{itemize}
        \item Intelligens ruhák
        \item Munka közben (katonaság, tűzoltóság)
        \item Sportolás közben (fitness)
    \end{itemize}
    \item Telemedicina
    \begin{itemize}
        \item Távdiagnózis
        \item Távmonitorozás
    \end{itemize}
    \item Autonóm robotok
    \begin{itemize}
        \item AUV, UAV, ROV
    \end{itemize}
\end{itemize}

\subsubsection{Távoló adatgyűjtő rendszerek általános működése}
\begin{itemize}
    \item Érzékelés
    \item Adattovábbítás
    \item Adatfeldolgozás, adatszűrés
    \item Adatelemzés
    \item Adatmegosztás, vizualizáció
    \item Az adatgyűjtő rendszer üzemeltetése
    \begin{itemize}
        \item Megjelenő hibák észlelése és korrigálása
    \end{itemize}
\end{itemize}

\subsection{Általános páciens monitorozó DAQ architektúra}
\customwidthimage{altpac}{15cm}

\subsection{Élettani paraméterek monitorozása extrém körülmények között}
\begin{center}
    \customwidthimage{eletext}{14cm}
    \customwidthimage{eletext2}{14cm}
\end{center}

\subsection{Üzemeltetési és távfelügyeleti rendszer}
\begin{itemize}
    \item Távmonitorozó rendszer
    \item Eseménykezelő rendszer
    \item Hibakezelő rendszer
    \item Monitorozási környezet
\end{itemize}