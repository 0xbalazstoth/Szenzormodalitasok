\section{Megvalósítás}

\subsection{Összeköttetés}
\customwidthimage{schematic}{15cm}

\subsection{Csomagok}
\subsubsection*{Websockets}
\begin{itemize}
    \item \href{https://github.com/Links2004/arduinoWebSockets}{https://github.com/Links2004/arduinoWebSockets}
\end{itemize}

\subsubsection*{DHT sensor}
\begin{itemize}
    \item \href{https://github.com/adafruit/DHT-sensor-library}{https://github.com/adafruit/DHT-sensor-library}
    \item \href{https://github.com/adafruit/Adafruit_Sensor}{https://github.com/adafruit/Adafruit$\_$Sensor}
\end{itemize}

\subsubsection*{GFX}
\begin{itemize}
    \item \href{https://github.com/adafruit/Adafruit-GFX-Library}{https://github.com/adafruit/Adafruit-GFX-Library}
    \item \href{https://github.com/adafruit/Adafruit_BusIO}{https://github.com/adafruit/Adafruit$\_$BusIO}
\end{itemize}

\subsubsection*{Display}
\begin{itemize}
    \item \href{https://github.com/adafruit/Adafruit_SH110X}{https://github.com/adafruit/Adafruit$\_$SH110X}
\end{itemize}

\subsection{BAUD}
A szenzorból származó adatok kiolvasása és megjelenítése a soros monitoron a 115200 baudos kommunikációs sebesség használatával történik. Ez a baud érték azt jelenti, hogy a rendszer másodpercenként 115200 bit adatot képes továbbítani.
\customwidthimage{baud}{10cm}

\subsection{setup()}
\textbf{Cél:} A rendszer hardver komponenseinek és kommunikációs protokolljainak inicializálása.

\textbf{Leírás:}
\begin{itemize}
  \item A soros port inicializálása 115200 baud sebességgel a debug üzenetekhez.
  \item A DHT22 szenzor aktiválása, ami a hőmérséklet és páratartalom méréséért felelős.
  \item Az OLED kijelző beállítása, ellenőrzése, hogy a kijelző helyesen van-e csatlakoztatva. Ha nem, a rendszer leáll.
  \item A kijelzőn megjelenő kezdő szöveg megjelenítése, ami a rendszer állapotának elsődleges vizuális visszajelzése.
\end{itemize}

\subsection{setupEEPROM()}
\textbf{Cél:} Az EEPROM-ból való Wi-Fi hitelesítő adatok olvasása és a hálózathoz való csatlakozás kísérlete.

\textbf{Leírás:}
\begin{itemize}
  \item Az EEPROM inicializálása és a korábban mentett Wi-Fi hitelesítő adatok (SSID és jelszó) olvasása.
  \item Wi-Fi hálózathoz való csatlakozás ezekkel az adatokkal.
  \item Ha a hálózati kapcsolat sikerült, a WebSocket szerver elindítása.
  \item Ha a kapcsolat nem jön létre, a rendszer hozzáférési pontként (AP módban) való indítása, hogy a felhasználók közvetlenül csatlakozhassanak és konfigurálhassák a hálózati beállításokat.
\end{itemize}

\subsection{connectToWiFi(const char* ssid, const char* password)}
\textbf{Cél:} Külön Wi-Fi hálózathoz való csatlakozás.

\textbf{Leírás:}
\begin{itemize}
  \item Megpróbál csatlakozni a megadott SSID-hez és jelszóval.
  \item A csatlakozási kísérlet során a kapcsolat állapotát jelző üzenetek megjelenítése.
  \item A kapcsolat létrejötte esetén az IP-cím kiírása.
\end{itemize}

\subsection{displayOledText(int x, int y, uint8\_t textSize, uint16\_t textColor, uint16\_t backgroundColor, String text)}
\textbf{Cél:} Szöveg megjelenítése az OLED kijelzőn.

\textbf{Leírás:}
\begin{itemize}
  \item A megadott paraméterek alapján szöveg kiírása a kijelzőre.
  \item A szöveg méretének, színének és pozíciójának beállítása.
\end{itemize}

\subsection{displayOledTextWrapped(int x, int \&y, uint8\_t textSize, uint16\_t textColor, uint16\_t backgroundColor, String text)}
\textbf{Cél:} Többsoros szöveg megjelenítése az OLED kijelzőn.

\textbf{Leírás:}
\begin{itemize}
  \item A szöveg megjelenítése több sorban, ha a szöveg hossza meghaladja a kijelző szélességét.
  \item Automatikus sortörés a szöveg megfelelő helyen történő tördelése érdekében.
\end{itemize}

\subsection{testWifi()}
\textbf{Cél:} Wi-Fi kapcsolat tesztelése.

\textbf{Leírás:}
\begin{itemize}
  \item Ellenőrzi, hogy a Wi-Fi kapcsolat aktív-e egy adott időintervallumon belül.
  \item Visszaadja a teszt eredményét (sikerült vagy sem).
\end{itemize}

\subsection{Értékek a kijelzőn}
\begin{itemize}
    \item A kijelző két részre lett bontva:
    \begin{enumerate}
        \item Felső rész, avagy a \textbf{status bar}.
        \item Kijelző maradék része, \textbf{egyéb információk} kijelzése.
    \end{enumerate}
\end{itemize}
\customwidthimage{display}{6cm}