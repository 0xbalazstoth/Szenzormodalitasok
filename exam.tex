\section{Vizsga kérdések}

\subsection{Szenzorok}
\subsubsection{Szenzor definíció}
\begin{itemize}
    \item A szenzor egy eszköz, ami fizikai ingereket (pl.: hőmérséklet, fény, nyomás) érzékel és ezeket mérhető jelekké alakítja.
    \item Ezek a jelek lehetővé teszik környezetünk változásainak észlelését és mérését. (pl.: ipar, egészségügy, környezetvédelem)
    \item Szenzorok lehetnek aktív vagy passzív típusúak, azaz saját energiát használó vagy a környezeti energiát felhasználó eszközök.
    \item Digitális adatot ad ki magából.
    \item A digitális adatot tovább adja küldeni vezetékes vagy vezeték nélküli csatornán.
\end{itemize}

\subsubsection{Szenzor kategóriák}
\begin{itemize}
    \item Hőmérséklet-szenzorok
    \begin{itemize}
        \item Környezet hőmérsékletének mérése, gyakran használják otthoni fűtési rendszerekben, autókban és ipari folyamatokban.
    \end{itemize}
    \item Gyorsulásmérők
    \begin{itemize}
        \item Gyorsulás/rezgés mértékét érzékelik, gyakran használják mobiltelefonok képernyő-orientációjának szabályozásához, járművek ütközésérzékelésére, épületek szeizmikus monitorozására.
    \end{itemize}
    \item Kémiai szenzorok
    \begin{itemize}
        \item Vegyületek mérése levegőben, vízben vagy egyéb közegben.
        \item Környezetszennyezés monitorozása, ipari folyamatok ellenőrzése, egészségügyben.
    \end{itemize}
    \item Hangfrekvencia-szenzorok
    \begin{itemize}
        \item Hanghullámokat érzékel, alkalmazhatóak biztonsági rendszerekben, hangfelismerésben és akusztikai elemzésben.
    \end{itemize}
\end{itemize}

\subsubsection{Szenzorok használati területei}
\begin{quote}
    A szenzorok javítják életünk minőségét, hatékonyságát és biztonságát.
\end{quote}
\begin{itemize}
    \item Okostelefonok és viselhető eszközök
    \begin{itemize}
        \item Gyorsulásmérők, giroszkópok, lépésszámlálás, automatikus fényerő-szabályozás.
    \end{itemize}
    \item Otthoni automatizálás és intelligens otthonok
    \begin{itemize}
        \item Hőmérséklet, fény, mozgás
    \end{itemize}
    \item Ipari automatizálás
    \begin{itemize}
        \item Üzemeltetés hatékonyságát növeli.
    \end{itemize}
    \item Egészségügy és orvostechnika
    \begin{itemize}
        \item Kémiai szenzorok, biometrikus szenzorok, diagnosztikák/kezelések monitorozása
    \end{itemize}
    \item Autóipar
    \begin{itemize}
        \item Ütközésérzékelők, parkolássegítő rendszerek
    \end{itemize}
    \item Biztonsági rendszerek
    \begin{itemize}
        \item Mozgásérzékelők, füstérzékelők, kamera szenzorok a veszélyhelyzetek azonosítására.
    \end{itemize}
\end{itemize}

\clearpage
\subsubsection{Szenzor problémák}
\begin{itemize}
    \item Kalibrációs problémák
    \begin{itemize}
        \item Következménye a pontatlanság lehet, kell a rendszeres kalibráció a pontos működéshez.
    \end{itemize}
    \item Környezeti hatások
    \begin{itemize}
        \item Extrém hőmérsékletek, por és egyéb környezeti tényezők befolyásolhatják a szenzorok teljesítményét.
    \end{itemize}
    \item Energiafogyasztás
    \begin{itemize}
        \item Akkumulátorral működő eszközöknél a szenzorok sokat fogyasztanak.
    \end{itemize}
    \item Inferencia és zaj
    \begin{itemize}
        \item Zajok torzíthatják a szenzorok által gyűjtött adatokat, ami pontatlansághoz vezet.
    \end{itemize}
    \item Technológiai korlátok (Például kommunikáció)
\end{itemize}

\subsubsection{Aktuátor definíció, példák}
\begin{quote}
    Az aktuátorok olyan eszközök, amik elektromos jelet fizikai műveletekké alakítanak át, az aktuátorok lehetnek mechanikus szerkezetek vagy bonyolultabb rendszerek.
\end{quote}
\begin{itemize}
    \item Elektromos motorok
    \begin{itemize}
        \item Elektromos energiát mechanikai mozgássá alakítanak át, járművek meghajtását teszik lehetővé például.
    \end{itemize}
    \item Hidraulikus aktuátorok
    \begin{itemize}
        \item Folyadék nyomásának növelésével/csökkentésével működnek, erős és precíz mozgatásra képesek, például építőipari gépekben.
    \end{itemize}
    \item Pneumatikus aktuátorok
    \begin{itemize}
        \item Sűrített levegőt használnak a mozgás előidézésére, például ahol gyors és ismétlődő mozgásra van szükség.
    \end{itemize}
\end{itemize}

\subsection{DAQ/Tradícionális vezérlés/DCS/SCADA/Monitoring/Vezérlőrendszerek}
\subsubsection{DIKW piramis}
\begin{quote}
    Tudáspiramis, egy modell, bemutatja hogyan alakulnak át az adatok értelmezhető és használható tudássá.
\end{quote}
\begin{enumerate}
    \item Adat (Nyers adatok kontextus nélkül (pl.: mérések adatai), kevés hasznos információt tartalmaznak.)
    \item Információ (Adatok kontextusba helyezése.)
    \item Tudás (Információkból következtetéseket lehet levonni.)
    \item Bölcsesség (Tudás alkalmazása, miértje.)
\end{enumerate}

\subsubsection{Tradícionális vezérlési rendszerek vs. Elosztott vezérlési rendszerek (DCS)}
\begin{itemize}
    \item Tradícionális vezérlési rendszerek
    \begin{itemize}
        \item Központosított architektúrára épül, ahol egy vagy több központi vezérlőegység végzi a folyamatok összes vezérlési és felügyeleti feladatát.
        \item Egyszerű, alacsony költségek
        \item Korlátozott skálázhatóság, a központosított vezérlés miatt nagyobb a rendszer kiesésének kockázata
    \end{itemize}
    \item Elosztott vezérlési rendszerek (DCS)
    \begin{itemize}
        \item Vezérlési folyamatokat moduláris egységek között osztja szét, amik kommunikálnak egymással egy közös hálózaton keresztül.
        \item Minden egyes szegmens a saját területéért felelős.
        \item Magasabb rendelkezésre állás, megbízhatóság, skálázhatóság
        \item Költséges, bonyolult
    \end{itemize}
\end{itemize}

\subsubsection{DAQ rendszer előnyök/hátrányok, alkalmazási területek}
\begin{quote}
    Adatgyűjtő rendszerek, lehetővé teszik fizikai jelenségek valós idejű monitorozását és analízisét. \\
    Egy tipikus DAQ rendszer szenzorokból áll, adatgyűjtő hardverből és szoftverből áll, amik összegyűjtik és feldolgozzák az adatokat.
\end{quote}
\begin{itemize}
    \item Előnyök (Rugalmas konfiguráció, pontos mérések, valós idejű adatfeldolgozás, automatizálás)
    \item Hátrányok (Költségek, technikai komplexitás, hardver és szoftverkompatibilitás, Karbantartás, frissítések)
    \item Alkalmazási területek (Tudományos kutatás, mérnöki tesztelés és fejlesztés, környezeti monitorozás, egészségügy)
\end{itemize}

\subsubsection{Elosztott vezérlési rendszerek (DCS)}
\begin{itemize}
    \item Előnyök (Magasabb rendelkezésre állás, megbízhatóság, skálázhatóság)
    \item Hátrányok (Költséges, bonyolult)
    \item Alkalmazási területek (Gyógyszeripar, erőművek, élelmiszeripar)
\end{itemize}

\subsubsection{SCADA rendszer}
\begin{itemize}
    \item Előnyök (Távoli felügyelet, valós idejű adatgyűjtés, megbízhatóság, automatizálás)
    \item Hátrányok (Komplexitás, költségek, karbantartás)
    \item Alkalmazási területek (Energiaipar, közlekedés, gyártás és automatizálás)
\end{itemize}

\clearpage
\subsection{DAQ}
\subsubsection{DAQ rendszer komponensei}
\begin{itemize}
    \item Szenzorok és érzékelők (Fizikai változások elektromos jelekké alakítja át)
    \item Jelkondicionáló áramkörök (Jeleket átalakítja, hogy azok megfelelőek legyenek)
    \item Adatgyűjtő eszközök - DAQ hardver (Előkészített analóg jeleket digitális formátumba konvertálja)
    \item Számítógép és interfész (DAQ hardvert számítógéphez kell csatlakoztatni, interfész lehet USB, PCI, PCIe, Ethernet és szoftveren keresztül kezeli az adatokat)
    \item Szoftver (LabVIEW, MATLAB)
\end{itemize}

\subsubsection{DAQ alkalmazási területek}
\begin{itemize}
    \item Tudományos kutatás, mérnöki tesztelés és fejlesztés, környezeti monitorozás, egészségügy
\end{itemize}

\subsubsection{Jelkondícionálás}
\begin{quote}
    A jelkondícionálás előkészíti az elektromos jeleket a digitális átalakításra és feldolgozásra, célja, hogy javítsa a jelek minőségét és növelje az adatgyűjtés pontosságát.
\end{quote}
\begin{itemize}
    \item Erősítés, szűrés, hőmérséklet-kompenzáció, lineárizáció, galvanikus leválasztás, jelalakítás
\end{itemize}

\subsubsection{Betegmonitorozó DAQ infrastruktúra}
\begin{quote}
    Beteg valós időben történő monitorozása.
\end{quote}
\begin{itemize}
    \item Szenzorok és érzékelők, jelkondícionáló áramkörök, adatgyűjtő eszközök, központi monitorozó állomás és hálózati infrastruktúra, szoftver és analitikai eszközök, adattárolás és archiválás
\end{itemize}

\subsubsection{DAQ rendszerek kihívásai}
\quote
\begin{itemize}
    \item Jelzaj és interferencia, nagy adatmennyiség kezelése, szenzor kalibráció és hőmérsékleti hatások, Adatbiztonság és adatvédelem, hardver és szoftverkompatibilitás, skálázhatóság és rugalmasság, Kezelhetőség 
\end{itemize}

\subsection{Elosztott vezérlési rendszerek (DCS)}
\subsubsection{DCS architektúra}
\begin{itemize}
    \item Vezérlési folyamatokat moduláris egységek között osztja szét, amik kommunikálnak egymással egy közös hálózaton keresztül.
    \item Minden egyes szegmens a saját területéért felelős.
    \item Moduláris
\end{itemize}

\subsubsection{DCS komponensei}
\begin{itemize}
    \item Folyamatvezérlők
    \item Operátori állomások (HMI)
    \item I/O modulok
    \item Kommunikációs hálózatok
    \item Mérnöki munkaállomások
    \item Adatarchiváló és elemző rendszer
    \item Biztonsági rendszerek
\end{itemize}

\subsubsection{DCS előnyök és hátrányok}
\begin{itemize}
    \item Előnyök (Magasabb rendelkezésre állás, megbízhatóság, skálázhatóság)
    \item Hátrányok (Költséges, bonyolult)
\end{itemize}

\subsubsection{DCS alkalmazási területek}
\begin{itemize}
    \item Gyógyszeripar, erőművek, élelmiszeripar
\end{itemize}

\subsection{Supervisory control and data acquisition - SCADA}
\subsubsection{SCADA architektúra, komponensei}
\begin{quote}
    Lehetővé teszik a nagy ipari és infrastrukturális folyamatok távoli monitorozását, vezérlését és automatizálását.
\end{quote}
\begin{itemize}
    \item Terepi eszközök
    \item Távközlési rendszerek
    \item RTU-k
    \item SCADA szerverek és számítógépek
    \item HMI
    \item Adatbázis és archiváló rendszerek
    \item Biztonsági komponensek
    \item Alkalmazási és szoftvereszközök
\end{itemize}

\subsubsection{SCADA előnyök/hátrányok}
\begin{itemize}
    \item Előnyök (Távoli felügyelet, valós idejű adatgyűjtés, megbízhatóság, automatizálás)
    \item Hátrányok (Komplexitás, költségek, karbantartás)
\end{itemize}

\subsubsection{SCADA alkalmazási területek}
\begin{itemize}
    \item Energiaipar, közlekedés, gyártás és automatizálás
\end{itemize}

\clearpage
\subsubsection{SCADA funkciói}
\begin{itemize}
    \item Távmérések, távjelzések fogadása
    \item Visszajelzés, adat vizualizáció
    \item Naplózás
    \item Riasztások (határérték és gradiens figyelés)
    \item Topológia analízis
    \item Távparancsadás
    \item Autentikáció és jogosultságkezelés
    \item Adattárolás
\end{itemize}

\subsubsection{Basic SCADA vs. Integrated SCADA vs. Networked SCADA}
\begin{itemize}
    \item Basic SCADA
    \begin{itemize}
        \item Alapvető távoli felügyelet és adatgyűjtés
        \item Korlátozott I/O kapacitás, egyszerű HMI és adatgyűjtés
        \item Kis teljesítményűek, egyszerű ipari folyamatok
    \end{itemize}
    \item Integrated SCADA
    \begin{itemize}
        \item Bonyolultabb, több funkciót integrálnak egyetlen koherens rendszerben.
        \item ERP rendszerek, komplex gyártási folyamatok, nagy létesítmények, vállalati szintű
    \end{itemize}
    \item Networked SCADA
    \begin{itemize}
        \item Széleskörű hálózati kapcsolatok, távoli elérés és vezérlés, adatmegosztás a létesítmények között, felhőalapú adattárolás és szolgáltatások
        \item Távvezetéki rendszerek, vízellátás, energiaelosztás, szétszórt infrastruktúra felügyelet
    \end{itemize}
\end{itemize}

\clearpage
\subsection{Biojel-gyűjtés (Biosignal acquisition)}
\begin{quote}
    A bioszignálok az élőlények testéből származó elektromos, mechanikai vagy más fizikai jelek, amik információt hordoznak az adott szervezet vagy szervrendszer állapotáról.
\end{quote}

\begin{itemize}
    \item Elektromos bioszignálok (Ideg és izomsejtek elektromos aktivitásából származnak)
    \begin{itemize}
        \item \textbf{EKG} (Elektrokardiogram), szív elektromos tevékenysége.
        \item \textbf{EEG} (Elektroenkefalográfia), Agy elektromos aktivitása.
        \item \textbf{EMG} (Elektromiográfia), izom elektromos aktivitása.
    \end{itemize}
    \item Mechanikai bioszignálok (Fizikai mozgások és változások)
    \begin{itemize}
        \item \textbf{Pulzushullám-velocitás}, az artériás rugalmasság mérésére szolgáló jel.
        \item \textbf{Spirometria}, légzés mechanikájának mérése.
    \end{itemize}
    \item Kémiai és biokémiai bioszignálok (Kémiai összetétel változásai)
    \begin{itemize}
        \item \textbf{Glükózszint-mérés}, vércukorszint mérés.
        \item \textbf{pH-mérés}, testfolyadék savasságának mérése.
    \end{itemize}
    \item Optikai és termikus bioszignálok (Fény és hő alapú jelek)
    \begin{itemize}
        \item \textbf{Pulzoximetria}, véroxigénszint mérés.
        \item \textbf{Testhőmérséklet-mérés}, a test belső hőmérsékletének mérési módjai.
    \end{itemize}
    \item Környezeti tényezők és egyéb mérések
    \begin{itemize}
        \item \textbf{Páratartalom és hőmérséklet}
        \item \textbf{Tartás és gyorsulás}
    \end{itemize}
\end{itemize}

\subsubsection{Szenzoradat kezelési folyamat}
\begin{enumerate}
    \item Data acquisition (DAQ) (egy/több szenzor)
    \item Adatkezelés (feldolgozás/szűrés)
    \item Tárolás, keresés
    \item Vizualizáció
    \item Megosztás
\end{enumerate}

\subsubsection{Mérési hibák (típusok/források)}
\begin{quote}
    Minden mérés tartalmaz hibákat!
\end{quote}
\begin{itemize}
    \item Rendszerszerű hibák
    \begin{itemize}
        \item \textbf{Kalibrációs hibák} (Mérőeszközök nem megfelelő kalibrálása)
        \item \textbf{Elektromos interferencia} (Környezeti elektromos berendezések zavarai)
        \item \textbf{Jelátviteli hiba} (Hosszú vagy rossz minőségű kábelezés)
    \end{itemize}
    \item Random hibák
    \begin{itemize}
        \item \textbf{Operátori hiba} (Emberi tényező, mint pl.: szenzor helytelen elhelyezése)
        \item \textbf{Fiziológiai zaj} (A testből származó nem kívánt jelek, pl.: izomzaj)
        \item \textbf{Mintavételi hiba} (Nem megfelelő mintavételi frekvencia alkalmazása)
    \end{itemize}
\end{itemize}

\clearpage
\subsubsection{Egyetlen vs. több szenzoros mérési problémák}
\begin{itemize}
    \item Egyetlen szenzoros mérési problémák
    \begin{itemize}
        \item \textbf{Korlátozott információszerzés}, egyetlen nézőpontból származó adatok korlátozott betekintést nyújtanak.
        \item \textbf{Hibatűrés hiánya}, egy szenzor meghibásodásánál nincs redundancia, ami az egész mérési folyamat kieséséhez vezethet.
        \item \textbf{Nagyobb kockázat a pontatlanságokra}, az adatok értelmezésekor
    \end{itemize}
    \item Több szenzoros mérési problémák
    \begin{itemize}
        \item \textbf{Adatkezelés és feldolgozás}, nagy adatmennyiség kezelése, tárolása, elemzése bonyolult, időigényes
        \item \textbf{Adatfúzió és integráció}, a különböző típusú és forrású adatok összeegyeztetése és integrálása technikai kihívást jelenthet.
        \item \textbf{Interferencia és koherencia}, a szenzorok közötti interferencia és az adatok koherenciájának hiánya torzíthatja az eredményeket.
    \end{itemize}
\end{itemize}

\clearpage
\subsubsection{Mérnöki kihívások a nagymennyiségű adatgyűjtésnél}
\begin{itemize}
    \item Általános kihívások
    \begin{itemize}
        \item P1 $\rightarrow$ Sok DAQ csomópont
        \item P2 $\rightarrow$ Sok szenzor (különböző típusú)
        \item P3 $\rightarrow$ Nagy adatmennyiség lehetőleg gyorsan átküldve
    \end{itemize}
    \item Kommunikációs probléma $\rightarrow$ P1 $\times$ P2 $\times$ P3
    \item Szoftver kihívások
    \begin{itemize}
        \item Valós idejű DAQ + előfeldolgozás + feldolgozás + megjelenítés (különböző tartományok)
        \item Komplex döntési helyzetek
        \item Online adatmenedzsment (megosztás + archiválás)
    \end{itemize}
    \item Hardver korlátok
    \begin{itemize}
        \item Energia problémák
        \item Kommunikációs hatótávolságok, adat multiplexálási problémák/idő, frekvencia
        \item Biztonság, megbízhatóság, használhatóság
    \end{itemize}
\end{itemize}

\clearpage
\subsection{Biojel-gyűjtés a sportban és az idősgondozásban}
\subsubsection{Sportmonitoring megoldások (aktivitás/teljesítmény) és problémák}
\begin{itemize}
    \item Viselhető eszközök (pulzusmérők, lépésszámlálók, GPS órák)
    \item Erő és mozgásérzékelők (gyorsulásmérők, giroszkópok)
    \item Problémák
    \begin{itemize}
        \item Adatpontosság és megbízhatóság (kalibrációs, szenzorhibák)
        \item Viselhetőség és kényelem
        \item Adatkezelés, adatbiztonság, adatvédelem
        \item Integráció más rendszerekkel (kompatibilitási problémák)
    \end{itemize}
\end{itemize}

\subsubsection{Környezet és lokáció monitoring (lokáció monitoring problémák)}
\begin{itemize}
    \item Környezet monitoring (Kamera, Sugárzásmérők, Levegőminőségmérők)
    \item Lokáció monitoring (GPS, Bluetooth/WiFi alapú rendszerek, GSM/2G/3G/4G/5G, IMU)
    \item Lokáció monitoring problémák
    \begin{itemize}
        \item \textbf{Pontatlanságok} (beltérben, beépített városi területeken)
        \item \textbf{Magas energiatakarékosság}
        \item \textbf{Integráció és kompatibilitás}
        \item \textbf{Jogi és etikai megfontolások}
    \end{itemize}
\end{itemize}

\subsubsection{Idősek monitorozása (problémák és megoldások)}
\begin{quote}
    Távoli betegmonitorozás (Diabétesz/Gyógyszeres/Hipertónia monitorozás)
\end{quote}
\begin{itemize}
    \item \textbf{Problémák} (Technológiai akadályok, fizikai és kognitív korlátok)
    \item \textbf{Megoldások} (Egyszerűsített interfészek, személyre szabott beállítások, oktatás/támogatások)
\end{itemize}

\subsubsection{Hipertónia/Magas vérnyomás monitoring}
\begin{itemize}
    \item \textbf{Módszerek}
    \begin{itemize}
        \item \textbf{Auszkultációs módszer} (Manuális vérnyomásmérés, stetoszkóp és egy manuális mandzsettát használnak a Korotkov-hangok hallgatására, ami lehetővé teszi a szisztolés és diasztolés vérnyomás mérését.)
        \item \textbf{Oszcillometriás módszer} (Automatizált módszer, mandzsetta automatikusan felfújódik, és az oszcillációkat méri, amikor lassan engedi le a levegőt, így meghatározva a vérnyomást.)
    \end{itemize}
    \item \textbf{Eszközök}
    \begin{itemize}
        \item Felső karos vérnyomásmérők, csuklós vérnyomásmérők, ujjvérnyomásmérők
    \end{itemize}
\end{itemize}

\clearpage
\subsubsection{Diabétesz/cukorbetegség monitorozása (módszerek, berendezések, CGM, mesterséges hasnyálmiriggyel kapcsolatos kérdések)}
\begin{itemize}
    \item \textbf{Monitorozási módszerek}
    \begin{itemize}
        \item \textbf{Ujjbegy tesztelés}, (hagyományos vércukorszint-mérés), kisméretű vércukormérő eszközök, csepp vért igényelnek.
        \item \textbf{Szenzorok}, (folyamatos glükózmonitorozás - CGM), bőr alá helyezett kis szenzorok folyamatosan mérnek.
    \end{itemize}
    \item \textbf{Berendezések} (Digitális glükózmérők, folyamatos monitorozó eszközök (CGM rendszerek))
    \item \textbf{CGM}
    \begin{itemize}
        \item Apró glükózérzkleő szenzort helyeznek a bőr alá, általában a hason vagy a felső kar hátsó részén.
        \item A szenzor méri a szöveti glükózszintet, és az információt továbbküldi egy eszközre.
        \item A rendszer átlagos vércukorértékeket rögzít 1-5 perces intervallumokban, folyamatos nyomon követés.
        \item Ujjbegy tesztelés és kalibráció szükséges a pontossághoz.
        \item A rendszer riasztásokat küld, ha a vércukorszint eléri vagy átlépi a beállított célértékeket.
        \item Inzulin "pumpával" kombinálva
        \item Mesterséges hasnyálmirigy, automatikus algoritmusok
        \item Inzulin, szénhidrát bevitele
    \end{itemize}
    \item \textbf{Mesterséges hasnyálmirigy problémák} (Rendszer összetettsége, automatizációs problémák)
\end{itemize}

\clearpage
\subsection{I2C és SPI (rövid hatótávolságú vezetékes kommunikáció)}
\subsubsection{I2C kommunikáció}
\begin{itemize}
    \item Integrált áramkörök összekapcsolására
    \item Busz alapú
    \item Kétvezetékes szinkron adatátviteli rendszer
    \item Két vezeték (SCL - órajel; SDA - adat)
    \item Alternatívák (SMBus/PMBus)
    \item Egy/több master és egy/sok slave
    \item Master csinál mindent
    \begin{itemize}
        \item Mindig a Master küldi az órajelet az SCL vonalra.
        \item Master kezdeményez adattranszfert
    \end{itemize}
    \item Slave fogadja az órajelet és válaszol ha kérdezik
    \item (Start) + cím +, egy bites vezérlő jel mutatja meg, hogy a megjelölt Slave-et a Master írni vagy olvasni kívánja.
    \item Master és Slave szerep felcserélhető.
    \item Felhúzó ellenállásokkal szokták kötni, de ez breakout boardoknál már általában be van építve.
\end{itemize}

\subsubsection{I2C előnyök/hátrányok, alkalmazási területek}
\begin{itemize}
    \item \textbf{Előnyök} (Alacsony vezetékszám, több eszköz támogatása, beépített konfliktuskezelés, rugalmas sebesség, egyszerű implementáció)
    \item \textbf{Hátrányok} (Korlátozott távolság, relatív lassúság, buszkonfliktusok több eszköznél, Master központúság)
    \item \textbf{Alkalmazási területek} (Beágyazott rendszerek, mobileszközök, elektronika)
\end{itemize}

\subsubsection{SPI komponensei}
\begin{quote}
    Serial Peripheral Interface, egy szinkron, soros adatkapcsolati protokoll, amit mikrokontrollerek és perifériás eszközök közötti gyors adatátvitelre használnak.
\end{quote}
\begin{itemize}
    \item \textbf{MOSI (Master Out Slave In)} (A Master eszköz adatkimeneti vonala, amin keresztül adatokat küld a sorrendben lévő eszközöknek.) 
    \item \textbf{MISO (Master In Slave Out)} (A sorrendben lévő eszköz adatkimeneti vonala, amin keresztül adatokat küld vissza a Master eszköznek.) 
    \item \textbf{SS (Slave Select)} (Eszköz aktiválása az adatátvitel idejére)
\end{itemize}

\subsubsection{SPI előnyök/hátrányok, alkalmazási területek}
\begin{itemize}
    \item \textbf{Előnyök} (Magas átviteli sebesség, teljes duplex kommunikáció, nincs címzés, egyszerű kommunikáció)
    \item \textbf{Hátrányok} (Több vezetékre van szükség, nem támogat busz megosztást, skálázhatóság, hiányzó beépített konfliktuskezelés)
    \item \textbf{Alkalmazási területek} (Adattároló eszközök, kijelzők, szenzorok, hálózati eszközök, digitális-analóg és analóg-digitális átalakítók)
\end{itemize}

\clearpage
\subsubsection{I2C vs. SPI}
\begin{table}[ht]
    \centering
    \begin{tabularx}{\textwidth}{X X X}
    \toprule
    \textbf{Jellemző} & \textbf{I2C} & \textbf{SPI} \\
    \midrule
    Sebesség & Lassabb, max 3.4 Mbps & Gyorsabb, több Mbps is lehet \\ \hline
    Vezetékek Száma & 2 (SDA, SCL) & Legalább 4 (MISO, MOSI, SCK, SS), több SS vezetékkel \\ \hline
    Címzés & Beépített címzési rendszer & Nincs beépített címzés, külön SS vezeték szükséges \\ \hline
    Adatátvitel Módja & Fél-duplex & Teljes duplex \\ \hline
    Komplexitás és Hardverigény & Egyszerűbb, kevesebb vezetékkel & Bonyolultabb, több vezetékkel \\ \hline
    Felhasználási Területek & Kisebb sebességű alkalmazások, érzékelők & Adatigényes alkalmazások, nagy sebességű interfészek \\
    \bottomrule
    \end{tabularx}
    \label{table:i2c_vs_spi}
\end{table}

\clearpage
\subsection{Vezeték nélküli kommunikációs technológiák}
\subsubsection{BAN (Body Area Network) hálózatok}
\begin{table}[ht]
    \centering
    \begin{tabularx}{\textwidth}{X X}
    \toprule
    \textbf{Jellemző} & \textbf{Információ} \\
    \midrule
    Hatótávolság & Néhány méter, a test közvetlen közelében \\ \hline
    Felhasználható Technológiák & Bluetooth Low Energy (BLE), ZigBee, NFC, Wi-Fi Direct \\ \hline
    Sebesség & ZigBee: alacsony, Wi-Fi Direct: magas \\ \hline
    Energiafogyasztás & Alacsony, hosszú akkumulátor-élettartam előnyös \\ \hline
    Felhasználási Területek & Egészségügyi monitorozás, fitness követés, okosruházat, személyes biztonság \\
    \bottomrule
    \end{tabularx}
    \label{table:ban_networks}
\end{table}

\subsubsection{PAN hálózatok}
\begin{table}[ht]
    \centering
    \begin{tabularx}{\textwidth}{X X}
    \toprule
    \textbf{Jellemző} & \textbf{Információ} \\
    \midrule
    Hatótávolság & 10 métertől 100 méterig \\
    \hline
    Felhasználható Technológiák & Bluetooth, Bluetooth Low Energy (BLE), ZigBee, NFC, Wi-Fi Direct \\
    \hline
    Sebesség & BLE: 1 Mbps, Wi-Fi: 250 Mbps-ig \\
    \hline
    Energiafogyasztás & Nagyon alacsonytól (BLE) közepesig (Wi-Fi) \\
    \hline
    Felhasználási Területek & Okoseszközök kapcsolata, adatátvitel, média streaming, okosotthon-vezérlés \\
    \bottomrule
    \end{tabularx}
    \label{table:pan_networks}
\end{table}

\clearpage
\subsubsection{LAN hálózatok}
\begin{table}[ht]
    \centering
    \begin{tabularx}{\textwidth}{X X}
    \toprule
    \textbf{Jellemző} & \textbf{Információ} \\
    \midrule
    Hatótávolság & Általában 100 métertől több kilométerig, bővíthető repeater-ekkel és bridge-ekkel \\
    \hline
    Felhasználható Technológiák & Ethernet (vezetékes), Wi-Fi (vezeték nélküli), PowerLine \\
    \hline
    Sebesség & Ethernet: akár 10/100/1000 Mbps (Gigabit), Wi-Fi: akár 600 Mbps-ig, attól függően, hogy melyik Wi-Fi szabványt használják \\
    \hline
    Energiafogyasztás & Vezetékes LAN esetében alacsony, Wi-Fi esetében változó, függ a használati módoktól és a hálózati forgalomtól \\
    \hline
    Felhasználási Területek & Irodai hálózatok, oktatási intézmények, otthoni hálózatok, internet-hozzáférés megosztása, fájlmegosztás, multimédia streaming \\
    \bottomrule
    \end{tabularx}
    \label{table:lan_networks}
\end{table}

\clearpage
\subsubsection{MAN hálózatok}
\begin{table}[ht]
    \centering
    \begin{tabularx}{\textwidth}{X X}
    \toprule
    \textbf{Jellemző} & \textbf{Információ} \\
    \midrule
    Hatótávolság & Általában 5 km-től 50 km-ig, városi vagy nagyvárosi területeken \\
    \hline
    Felhasználható Technológiák & Ethernet, ATM, Frame Relay, MPLS, WiMAX, LTE \\
    \hline
    Sebesség & Több Mbps-től Gbps-ig terjedő sebességek, a technológiától függően \\
    \hline
    Energiafogyasztás & Függ a használt infrastruktúrától és technológiától, jellemzően magasabb, mint a LAN-oknál \\
    \hline
    Felhasználási Területek & Városi hálózatok összekötése, internet-szolgáltatók, nagyvárosi területi hálózatok, egyetemi kampuszok \\
    \bottomrule
    \end{tabularx}
    \label{table:man_networks}
\end{table}

\clearpage
\subsubsection{WAN hálózatok}
\begin{table}[ht]
    \centering
    \begin{tabularx}{\textwidth}{X X}
    \toprule
    \textbf{Jellemző} & \textbf{Információ} \\
    \midrule
    Hatótávolság & Több száz vagy ezer kilométer, nemzetközi és kontinentális távolságok \\
    \hline
    Felhasználható Technológiák & MPLS, Frame Relay, ATM, Ethernet WAN, VPN, Satellite, LTE/5G \\
    \hline
    Sebesség & Különböző, Mbps-tól Gbps-ig, a használt technológiától és infrastruktúrától függően \\
    \hline
    Energiafogyasztás & Jellemzően magas, különösen a hosszú távú infrastruktúrák és nagy teljesítményű eszközök miatt \\
    \hline
    Felhasználási Területek & Nemzetközi vállalati hálózatok, távoli adatközpontok összekapcsolása, internet-hozzáférés, távoli munka \\
    \bottomrule
    \end{tabularx}
    \label{table:wan_networks}
\end{table}

\subsection{USB}
\subsubsection{USB architektúra/komponensei, eszköz kategóriák}
\begin{itemize}
    \item \textbf{Architektúra építőelemei}
    \begin{itemize}
        \item Host Controller, USB portok, USB eszközök, USB hubok, adatátviteli protokollok
    \end{itemize}
    \item \textbf{Eszköz kategóriák}
    \begin{itemize}
        \item Tároló eszközök, beviteli eszközök, kommunikációs eszközök, audio és videó eszközök, nyomtatók és szkennerek, okos készülékek
    \end{itemize}
\end{itemize}

\clearpage
\subsubsection{USB előnyök/hátrányok, alkalmazási területek}
\begin{itemize}
    \item \textbf{Előnyök}, univerzalitás, energiaellátás, adatátvitel, portabilitás
    \item \textbf{Hátrányok}, biztonsági kockázatok, port korlátozottság, kompatibilitási problémák
    \item \textbf{Alkalmazási területek}, multimédia, hálózati alkalmazások, tárolóeszközök, perifériák, mobileszközök
\end{itemize}

\subsubsection{USB csatlakozók, OTG (On the Go)}
\begin{itemize}
    \item \textbf{USB csatlakozók}
    \begin{itemize}
        \item USB-A, USB-B, Mini-USB, Micro-USB, USB-C
    \end{itemize}
    \item \textbf{USB On the Go (OTG)}
    \begin{itemize}
        \item Kiegészítő szabvány, kétirányú kommunikáció, hordozhatóság, komptakt csatlakozó
    \end{itemize}
\end{itemize}

\subsubsection{USB Enumeration}
\begin{itemize}
    \item Kommunikáció a perifériával, feltérképezi, hogy milyen eszköz, driver betöltés, azonosító ID hozzárendelés, az áramfelvétel szabályozása.
\end{itemize}

\subsubsection{USB. RS-232 (serial)}
\begin{table}[ht]
    \centering
    \begin{tabularx}{\textwidth}{X X}
    \toprule
    \textbf{Jellemző} & \textbf{USB vs. RS-232} \\
    \midrule
    Sebesség & USB: Akár 10 Gbps (USB 3.1) vs. RS-232: Maximum 115,200 bps \\
    \midrule
    Csatlakozók & USB: Több típus, beleértve A, B, Mini, Micro, C vs. RS-232: Általában D-sub 9 vagy 25 pines csatlakozó \\
    \midrule
    Kábelezés & USB: Maximum 5 méter vs. RS-232: Akár 15 méter vagy több, jellemzően kisebb adatsebesség mellett \\
    \midrule
    Energiaellátás & USB: Eszközök tápellátása lehetséges a buszon keresztül vs. RS-232: Nem biztosít tápellátást \\
    \midrule
    Alkalmazási terület & USB: Széles körű, beleértve számítástechnikát, mobil eszközöket, tárolót vs. RS-232: Ipari vezérlés, régebbi eszközök kommunikációja \\
    \bottomrule
    \end{tabularx}
    \label{table:usb_vs_rs232}
\end{table}

\clearpage
\section{Fogalmak}
\begin{itemize}
    \item \textbf{Szenzor (sensor)}: A szenzor egy eszköz, ami fizikai ingereket (hőmérséklet, fény, nyomás) érzékel és mérhető jelekké alakítja.
    \item \textbf{Modalitás (modality)}: Érzékelés módja (látás, tapintás, hallás)
    \item \textbf{Okos város (smart city)}: Olyan város, ami az infokommunikációs technológiák segítségével hatékonyabban és fenntarthatóbban üzemelteti a szolgáltatásait. (közlekedés, energiaellátás)
    \item \textbf{Okos mezőgazdaság (smart agriculture)}: Az infokommunikációs technológiákat használó mezőgazdaság, ami a termelést optimalizálja és a hatékonyságot növeli.
    \item \textbf{IoT}: Olyan eszközök hálózata, amik az interneten keresztül kommunikálnak egymással és adatokat gyűjtenek.
    \item \textbf{Távgyógyászat (telemedicine)}: Távoli gyógyítás, ahol az orvos és a beteg nem ugyanazon a helyen tartózkodik.
    \item \textbf{ICT}: Információs és kommunikációs technológiák gyűjtőneve, internet, mobiltelefon, számítógép.
    \item \textbf{Precíziós mezőgazdaság (precision farming)}: GPS és szenzorok használatával optimalizálja a mezőgazdasági termelést.
    \item \textbf{Viselhető elektronikai eszközök (wearable devices)}: Viselhető elektronikai eszközök, okosórák, fitneszórák
    \item \textbf{Összekapcsolt eszközök (connected devices)}: Interneten vagy belső hálózaton összekapcsolt eszközök, amik adatot cserélnek.
    \item \textbf{Átalakító (transducer)}: Fizikai jelet elektromos jellé alakít.
    \item \textbf{A/D konverter (A/D converter)}: Analóg jelet digitális jellé alakít.
    \item \textbf{Zavaró jel (signal noise)}: A hasznos jelhez nem kapcsolódó zavaró jel.
    \item \textbf{Bináris kód (binary code)}: 0 és 1-es számjegyekkel ábrázolt információ.
    \item \textbf{Aktuátor (actuator)}: Elektromos jelet mechanikai mozgássá alakít.
    \item \textbf{Jel kondícionálás (signal conditioning)}: A jel erősségének, zajszintjének és formátumának az érzékelőhöz vagy továbbításához való igazítása.
    \item \textbf{Jel skálázása (signal scaling)}: A jel erősségének arányos módosítása egy kívánt tartományba.
    \item \textbf{Amplifikáció/erősítés (amplification)}: A jel erősségének növelése.
    \item \textbf{Linearizáció (linearization)}: A nemlineáris jel lineáris közelítésbe történő alakítása.
    \item \textbf{Kompenzáció (compensation)}: A mérésihibák kiegyenlítése.
    \item \textbf{Szűrés (filtering)}: A nem kívánt jelek eltávolítása a jelből.
    \item \textbf{Csökkentés (attenuation)}: A jel erősségének csökkentése.
    \item \textbf{Gerjesztés (excitation)}: Egy rendszer bemenő jellel való ellátása.
    \item \textbf{Elosztott diagnosztika (distributed diagnostic)}: A rendszer különböző pontjain történő hibafeltárás.
    \item \textbf{Elosztott hozzáférés (distributed access)}: Több eszköz egy kommunikációs csatornát használ közösen.
    \item \textbf{Fieldbus}: Digitális kommunikációs hálózat terepi eszközök és vezérlők között.
    \item \textbf{Telemetria (telemetry)}: Távoli adatgyűjtés és átvitel, szenzor adatainak továbbítása.
    \item \textbf{Elosztott vezérlő rendszer (Distributed Control System - DCS)}: Vezérlési folyamatokat moduláris egységek között osztja szét, amik kommunikálnak egymással egy közös hálózaton keresztül.
    \item \textbf{Supervisory control and data acquisition (SCADA)}: Lehetővé teszik a nagy ipari és infrastrukturális folyamatok távoli monitorozását, vezérlését és automatizálását.
    \item \textbf{DAQ}: Adatgyűjtő rendszerek, lehetővé teszik fizikai jelenségek valós idejű monitorozását és analízisét.
    \item \textbf{MTU - Master Terminal Unit}: Fő terminálegység, távvezérlő rendszerekben a kommunikációt irányítja.
    \item \textbf{Front End processor}: Előfeldolgozó egység, adatgyűjtő és előkészítő modul.
    \item \textbf{Biztonsági szerver (safety server)}: Ipari rendszerek védelmét biztosítja.
    \item \textbf{HMI - Human Machine Interface}: Felhasználói felület, ami összekapcsolja az embert egy géppel, rendszerrel, eszközzel.
    \item \textbf{MMI - Machine Machine Interface}: Információ csere, RFID, Bluetooth, telemetria
    \item \textbf{RTU - Remote Terminal Unit}: Távoli terminál egység, ami adatokat gyűjt és továbbít egy vezérlőrendszerhez.
    \item \textbf{PLC - Programmable Logic Controller}: Programozható logikai vezérlő, ami automatizálja az ipari folyamatokat.
    \item \textbf{IED - Intelligent Electronic Device}: Intelligens elektronikus eszköz, ami méri és elemzi a villamosenergia-hálózat adatait.
    \item \textbf{Trending}: Adatok időbeli változásának nyomon követése és elemzése.
    \item \textbf{API}: Alkalmazásprogramozási felület, ami lehetővé teszi a szoftverek közötti kommunikációt.
    \item \textbf{Biosignal}: Elektromos jel, amit a szervezet élő szövetei bocsátanak ki.
    \item \textbf{Élő organizmus (living organism)}: Biológiai organizmus, ami képes önállóan fenntartani a homeosztázist, szaporodni és fejlődni.
    \item \textbf{Anyagcsere (metabolism)}: Energia előállítása és a szervezet működéséhez szükséges anyagok létrehozása, lebontása és átalakítása.
    \item \textbf{Homeosztázis (homeostasis)}: Állandó belső környezet fenntartása a szervezetben.
    \item \textbf{Ingerek (stimuli)}: Ingerek, amik a szervezetet reakcióra késztetik.
    \item \textbf{Invazív mérés (invasive measurement)}: Testbe kell hatolni (vérvétel).
    \item \textbf{Nem-invazív mérés (non-invasive measurement)}: Nem igényel testbe való behatolást (röntgen).
    \item \textbf{Nyquist-Shannon mintavételi tétel (Nyquist-Shannon sampling theorem)}: Meghatározza a maximális jelhűséget adott mintavételi frekvenciánál.
    \item \textbf{AAMI}: Amerikai Orvosi Műszerezés Fejlesztési Társaság, egészségügyi technológia fejlesztésével foglalkoznak
    \item \textbf{BHS}: Brit Hipertónia Társaság, magas vérnyomás megelőzést, diagnosztizálást segítik elő.
    \item \textbf{Távoli monitorozás (remote monitoring)}: Távműködésű megfigyelés, ahol az adatokat távolról gyűjtik.
    \item \textbf{Változékonyság (variability)}: Ingadozás
    \item \textbf{Fehér köpeny szindróma (white coat syndrome)}: Emelkedett pulzus és/vagy vérnomyás orvosi vizsgálatkor
    \item \textbf{Korotkov hangok}: A vérnyomásméréskor hallható hangok, amik alapján állapítják meg a vérnyomás értékét.
    \item \textbf{Gyógyszeres monitoring (medication monitoring)}: A gyógyszerek hatásának és mellékhatásának monitorozása.
    \item \textbf{Magas vérnyomás (hypertension)}: Hipertónia, magasabb az artériák falára gyakorolt nyomás az átlagosnál.
    \item \textbf{Szisztolés nyomás-erő (systolic pressure-force)}: A szisztoléban a szívizom összehúzódik és vért pumpál a keringésbe.
    \item \textbf{Diasztolés nyomás-erő (diastolic pressure-force)}: A diasztolé alatt a szív ellazul és újratöltődik vérrel.
    \item \textbf{Auszkultációs módszer (auscultation method)}: Manuális vérnyomásmérés, stetoszkóp és egy manuális mandzsettát használnak a Korotkov-hangok hallgatására, ami lehetővé teszi a szisztolés és diasztolés vérnyomás mérését.
    \item \textbf{Oszcillometriás módszer (oscillometry method)}: Automatizált módszer, mandzsetta automatikusan felfújódik, és az oszcillációkat méri, amikor lassan engedi le a levegőt, így meghatározva a vérnyomást.
    \item \textbf{PPG-alapú mérés (PPG-based measurement)}: A hajszálerek térfogat változását regisztrálja.
    \item \textbf{Tonometria módszer (tonometry method)}: Mechanikai tapintófejjel rögzítésre kerül a csuklóartéria pulzálása.
    \item \textbf{BPM}: Pulzus, szaporaság, az 1 perc alatt mért lüktetések száma.
    \item \textbf{LDL}: Low density lipoprotein, rossz koleszterin, ami az érfalban lerakódva érelmeszesedést okozhat.
    \item \textbf{HDL}: High density lipoprotein, jó koleszterin, ami eltávolítja a koleszterint az érfalról.
    \item \textbf{Laktát monitoring (lactat monitoring)}: A vér tejsavszintjének mérése.
    \item \textbf{Koleszterinszint (cholesterol level)}: A koleszterin egy zsírszerű anyag, ami a szervezetünkben létfontosságú szerepet tölt be.
    \item \textbf{Trygliceridek szintje (Tryglicerides level)}: Az ember testzsír fő összetevői.
    \item \textbf{Glükóz (glucode)}: Egyszerű szénhidrát, szölőcukor
    \item \textbf{Inzulin receptor}: A sejtmembránon található kapcsoló, amihez az inzulin kötődik, hogy a glükóz bejusson a sejtbe.
    \item \textbf{Inzulin rezisztancia (inzulin resistance)}: A sejtek nem reagálnak megfelelően az inzulinra, emiatt a glükóz nem jut be a sejtekbe és a vércukorszint emelkedik.
    \item \textbf{Cukorbetegség (diabetes mellitus)}: A vércukorszint kóros emelkedése, ami inzulinhiány vagy inzulinrezisztancia következménye.
    \item \textbf{2-es típusú cukorbetegség}: Kevésbé súlyos mint az 1-es típusú, életmódváltoztatással és súlycsökkentéssel kezelhető. A szervezet nem hasznosítja hatékonyan az inzulint.
    \item \textbf{1-es típusú cukorbetegség}: Rendszeres inzulin adagolás szükséges mert a szervezet nem képes elegendő inzulint termelni!
    \item \textbf{Gestitational diabetes}: Terhességi cukorbetegség
    \item \textbf{HbA1c}: Glikált hemoglobin, akkor alakul ki, amikor a hemoglobin a vérben lévő glükózhoz kapcsolódik és glikálódik.
    \item \textbf{Lándzsás eszközök (lancets)}: Ujjbegybe szúrt vérvevők
    \item \textbf{Tesztcsík (test stripe)}: Színelváltozással leolvasható róla a glükózszint.
    \item \textbf{Fotometrikus módszer (photometric method)}: Szín-összehasonlítással olvassuk le a glükózszintet.
    \item \textbf{Amperometrikus módszer (amperometric method)}: Elektromos áramot vezetnek át a vett mintán és az áramerősségnek a segítségével mérik meg a glükóz mennyiséget.
    \item \textbf{CGM}: Apró glükózérzékelő szenzort helyeznek a bőr alá, általában a hason vagy a felső kar hátsó részén.
    \item \textbf{Mesterséges hasnyálmirigy (artificial pancreas)}: Automatikusan szabályozza a vércukorszintet inzulin beadásával és a glükagon kiválasztásával.
    \item \textbf{Inzulin bevitel (insulin intake)}: A vércukorszint szabályozására szolgáló hormon bevitele, injekció, inzulinpumpa vagy inhalátor segítségével.
    \item \textbf{Pirulaadagoló (pill dispenser)}: Tabletták, kapszulák automatizált adagolása beállított időpontokban.
    \item \textbf{Szívinfarktus (heart infarct)}: A szívizom vérellátását biztosító koszorúerek elzáródása, ami szívizom károsodáshoz vezet.
    \item \textbf{Sztrók (stroke)}: Az agy vérellátását biztosító erek elzáródása vagy megrepedése ami agyi károsodáshoz vezet.
    \item \textbf{AAL - Ambient Assisted Living}: Intelligens technológiák használata az idősek és betegek otthoni gondozására.
    \item \textbf{Digitális sztetoszkóp (digital stethoscope)}: Elektronikus eszköz a szív hangjainak hallgatására és rögzítésére.
    \item \textbf{ECG}: A szív elektromos aktivitásának grafikus ábrázolása.
    \item \textbf{MET - Calorie/Metabolic equivalent}: Ez egy olyan egység, amely az adott tevékenység intenzitását fejezi ki a pihenési anyagcsere által felhasznált energia mennyiségével összehasonlítva.
    \item \textbf{S-T emelkedés (S-T elevation)}: Az EKG-n az S-T hullám emelkedése.
    \item \textbf{EEG}: Az agy elektromos aktivitásának mérése.
    \item \textbf{EMG}: Az izmok elektromos aktivitásának mérése.
    \item \textbf{Apnoe}: Az apnoé során a légzés szünetel vagy jelentősen lelassul egy bizonyos időszakban. Típusai például: alvási apnoé vagy újszülött apnoé.
    \item \textbf{IMU}: Inerciális mérési egység, ami a mozgás mérésére szolgál.
    \item \textbf{Peak flow meter}: Csúcsáramlásmérő.
    \item \textbf{Spirometer}: Tüdőkapacitás és légzési funkciók mérésére szolgál.
    \item \textbf{Testösszetétel-mérők (Body Composition Monitors)}: A test zsír-, izom-, és víztartalmának mérésére szolgál.
    \item \textbf{RS-232}: Serial kommunkációs protokoll, régi eszközök használják.
    \item \textbf{SPI}: Soros perifériás interfész, adatátvitelre használják mikrovezérlők és perifériák között.
    \item \textbf{I2C}: Két eszköz kommunikációjára szolgál.
    \item \textbf{UART}: Univerzális aszinkron vevő-adó, soros adatátvitelre használják.
    \item \textbf{USART}: Univerzális szinkron/aszinkron vevő-adó, az UART továbbfejlesztett változata.
    \item \textbf{CAN}: Controller Area Network, ipari automatizálásban és járművekben használt kommunkációs protokoll.
    \item \textbf{1wire}: Egyszerű, egyvezetékes kommunikációs protokoll.
    \item \textbf{USB}: Universal Serial Bus egyszabványos csatlakozó, amivel perifériákat és tartozékokat csatlakoztathatunk számítógépekhez és más eszközökhöz.
    \item \textbf{Testkommunikáció (body communication)}: A test kommunikációja a testen viselt eszközökön keresztül történik, melyek érzékelik a testmozgást, a fiziológiai paramétereket és a környezeti feltételeket.
    \item \textbf{Textilhuzalok/vezető fonalak (Textile wires/conductive yarns)}: A textiliparban használt vezető szálak elektromos jelek továbbítására alkalmasak, és integrálhatók ruházatba és kiegészítőkbe.
    \item \textbf{8N1N}: Soros kommunikáció során alkalmazott karakterek formátumát írja le: 8 bit adat, Nincs paritás bit, 1 állapot bit, Nincs flow control
    \item \textbf{DTE}: Data Terminal Equipment (DTE) egy olyan végberendezés, amely a felhasználói információkat jelekké alakítja át, vagy a fogadott jeleket visszaalakítja felhasználható információvá.
    \item \textbf{DCE}: Data Communications Equipment (DCE) olyan számítógépes hardvereszközökre utal, amelyeket adatforrás és a célállomás közötti kommunikációs hálózati kapcsolatok létrehozására, fenntartására és lezárására használnak.
    \item \textbf{baud}: Szimbólumsebesség mértékegysége. A szimbólumsebesség azt adja meg, hogy másodpercenként hány jelváltozás történik az adatátviteli csatornán.
    \item \textbf{JTAG}: Joint Test Action Group egy szabványos soros interfész, amelyet általában a digitális eszközök fejlesztésében, tesztelésében és karbantartásában használnak.
    \item \textbf{MOSI : Master Output, Slave Input}: SPI kommunikációs protokoll esetében a Master küldi az adatokat, és a Slave fogadja őket ezen a vonalon keresztül.
    \item \textbf{MISO : Master Input, Slave Output}: SPI kommunikációs protokoll esetében a Slave küldi az adatokat, és a Master fogadja őket ezen a vonalon keresztül.
    \item \textbf{SS : Slave Select}: Az SS (Slave Select) jel az SPI kommunikációban azonosítja és aktiválja az adott szolga eszközt, lehetővé téve a mester számára, hogy vele kommunikáljon.
    \item \textbf{SDA (serial data)}: I2C kommunikációs protokoll esetében "Serial Data Line" rövidítése. Ez a vonal a tényleges adatok átvitelére szolgál.
    \item \textbf{SCL (serial clock)}: I2C kommunikációs protokoll esetében a "Serial Clock Line" rövidítése. Ez a vonal szabályozza az adatok cseréjének időzítését az "SDA" vonalon keresztül, segítve a kommunikáló eszközöknek szinkronizálni az adatok küldését és fogadását.
    \item \textbf{FIFO}: First-In, First-Out adatstruktúra, melyben az elsőként beérkező elemeket tárolja el, és azokat az elsőként veszi ki a struktúrából. Pl: stack
    \item \textbf{SCSI}: A "Small Computer System Interface" egy olyan interfész szabvány, amely lehetővé teszi a számítógépek és perifériák közötti adatátvitelt.
    \item \textbf{NRZI}: "Non-Return-to-Zero Inverted" egy digitális jelkódolási technika, azaz egy bit értékét az áramkör az adatjelben hozott változás határozza meg.
    \item \textbf{ACK}: "Acknowledgment" egy jelzés vagy válasz, amelyet egy fogadó eszköz küld egy küldő eszköznek, hogy megerősítse vagy visszaigazolja, hogy sikeresen fogadta és megértette az átvitt adatokat vagy üzenetet. Használat például: például az Ethernet, a TCP/IP, az USB, vagy a soros kommunikáció során.
    \item \textbf{OTG}: On the go, Kiegészítő szabvány, kétirányú kommunikáció, hordozhatóság, komptakt csatlakozó.
    \item \textbf{BAN}: Body Area Network, a testen viselt vezeték nélküli szenzorok hálózata.
\end{itemize}