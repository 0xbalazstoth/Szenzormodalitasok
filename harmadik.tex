\section{Téma 4}

\subsection{Elektromos jelek gyűjtése a testből/ről}
\begin{itemize}
    \item EKG $\rightarrow$ ElectroCardioGraphy
    \begin{itemize}
        \item Non-invazív szívvizsgáló eljárás, ami a szív elektromos jelenségeit vizsgálja, a szívizom-összehúzódásakor keletkező elektromos feszültség regisztrálásával.
    \end{itemize}
    \item EMG $\rightarrow$ ElectroMyoGraphy
    \item EEG $\rightarrow$ ElectroEncephaloGraphy
    \item ECoG/ECG $\rightarrow$ ElectroCOrticoGraphy
\end{itemize}

\subsection{Elektrokardiográf (EKG)}
\begin{itemize}
    \item A szív elektromos jeleit vizsgáló eljárás (a szívizom-összehúzódásakor
    keletkező elektromos feszültséget regisztrálja).
    \item Az EKG működési elve: a szívben lejátszódó elektromos folyamatok a
    test felszínén is jól érzékelhetőek, hiszen az emberi test jó vezető. Az
    apró elektromos változásokat felerősítve a jelek időben ábrázolhatók,
    így alakult ki az EKG-görbe.
    \item Az elektromos ingerületet a test felszínére helyezett elektródákkal
    lehet érzékelni. Az EKG hullám szabályos görbét ír, melynek egyedi
    alakja és sajátosságai vannak.
    \item Az EKG felfedezése és az első maihoz hasonló elven működő EKG
    készülék
\end{itemize}

\clearpage
\subsection{EKG}
\begin{itemize}
    \item Az első EKG-berendezések három ponton, a két kézen
    és a bal lábon mérték az elektromos változásokat.
    (1912-ben Einthoven bemutatja az Einthovenháromszöget)
    \item A három pont (a törzshöz való csatlakozásukat
    számolva) egy szabályos háromszöget ad ki,
    középpontjában a szívvel. Ennek a három pontnak az
    egymáshoz viszonyított feszültségértékeiből mérhető a
    feszültség változásának három külön értéke, amit
    római számokkal jelölnek.
\end{itemize}

\subsection{EKG bemenet funkcionális felépítése}
\customwidthimage{ekg}{13cm}

\clearpage
\subsection{EKG görbe}
\customwidthimage{gorbe}{7cm}

\begin{itemize}
    \item Egy szabályos EKG-felvételen öt csúcsot lehet megkülönböztetni, ezek a P, Q, R,
    S, T betűkkel vannak jelölve. Az egyes csúcsok megfelelnek bizonyos
    eseményeknek a szívben (depolarizáció - elektromos kisülés vagy repolarizáció
    - elektromos újratöltődés).
    \item \textbf{P} $\rightarrow$ - ingerület a szinusz csomóban (a pitvaron áthaladó elektromos impulzust, a
    pitvar összehúzódását jelzi)
    \item \textbf{Q} $\rightarrow$ az ingerület kezdete a kamrákban, ez az apró negatív csúcs gyakran nem is
    látható, ha nagyon megnövekszik, az infarktust jelezhet
    \item \textbf{R} $\rightarrow$ a legnagyobb csúcs a kamrákon végigterjedő ingerületet mutatja
    \item \textbf{S} $\rightarrow$ ez a negatív csúcs a kamrán végigfutó ingerület végét jelzi
    \item \textbf{T} $\rightarrow$ a kamra repolarizációját mutatja
    \item \textbf{U} $\rightarrow$ a normális görbén nem vagy csak alig látható, kóros állapotokban, például
    káliumhiány esetén látványosan jelenik meg
    \item \textbf{PQ intervallum} pitvar - kamrai átvezetést jelent
    \item Minden egyes normális szívverés tartalmaz egy P-hullámot és QRSkomplexumot és egy T-hullámot.
\end{itemize}

\clearpage
\subsection{EKG mérés régen és ma}
\begin{itemize}
    \item Kórházi EKG
    \item Transzfónikus EKG-k (Mentőszolgálat)
    \item Mobil EKG-k
    \begin{itemize}
        \item 66*59*17mm, 50 g, 1-2-3-5-12 csatorna
        \item Mintavételi frekvencia:150-300-600 Hz
        \item Időkorlátos/non-stop Holter monitor
    \end{itemize}
    \item Viselhető eszközök
\end{itemize}

\subsection{12 elvezetéses EKG – kórházi használat}
\begin{minipage}[t]{0.7\textwidth}
    \vspace{0pt}
    \begin{itemize}
        \item A 12 elvezetés EKG során 10 elektródát használunk. Elhelyezkedésük a
        testen a következő
        \begin{itemize}
            \item RA (right arm): a jobb kézen, csontos kiemelkedések kerülendők.
            \item LA (left arm ): a bal kézen ugyan oda ahol az RA van, csontos
            kiemelkedések kerülendők, azonos időben alkalmazandó.
            \item RL (right leg): jobb lábon, csontos kiemelkedések kerülendők.
            \item LL (left leg ): bal lábon, csontos kiemelkedések kerülendők,
            ugyanoda ahová az RL került, azonos időben alkalmazandó.
            \item V1: a negyedik bordaközhöz (4. és 5. borda között) a szegycsont
            jobb oldalára.
            \item V2: a negyedik bordaközhöz (4. és 5. borda között) a szegcsont bal
            oldalára.
            \item V3: a V2 és V4 közé.
            \item V4: az ötödik bordaközhöz (5. és 6. borda között) a középső
            klavikuláris régióba (calvicula - kulcscsont).
            \item V5: a V4-el horizontálisan a line axillaris anterior vonalába.
            \item V6: a V4 és a V5 elvezetésekkel horizontálisan, a közép hónalji
            vonalon.
        \end{itemize}
    \end{itemize}
\end{minipage}%
\hfill
\hspace{1cm}
\begin{minipage}[t]{0.3\textwidth}
    \vspace{0pt}
    \customwidthimage{_12}{6cm}
\end{minipage}

\clearpage
\subsection{Mobil EKG rendszerek a gyakorlatban}
\begin{itemize}
    \item Szív és keringési rendszer monitorozás
    \begin{itemize}
        \item Terápia követés
        \item Távoli páciensmonitorozás-gondozás
        \item Ambuláns EKG holter monitorozás 24/7!
        \item Edzés/sportmozgás monitorozás (egyéni és csapatsportra) 
        \begin{itemize}
            \item Terheléses EKG (pl.: 5 csatorna)
        \end{itemize}
    \end{itemize}
\end{itemize}

\subsection{EKG monitorok}
\begin{itemize}
    \item Orvosi minőség
    \begin{itemize}
        \item CardioBlue (esemény monitor)
        \item Wiwe
        \item Savvy
    \end{itemize}
\end{itemize}

\subsection{Mobil - viselhető EKG monitorok}
\begin{table}[h!]
    \centering
    \footnotesize
    \begin{tabularx}{\textwidth}{|X|X|X|X|}
    \hline
    \textbf{Model type} & \textbf{Manufacturer} & \textbf{Sensor type} & \textbf{Connection type} \\ \hline
    Bioharness & Zephyr Technology Ltd. & < DAQ harness: pulse,posture,RR,Heart rate & Wireless (BTv2) \\ \hline
    Quardio (Consumer electronic grade) & < Quardio Inc. & Mobile ECG , GSR, temperature, pulse, breathing activity, 1 day & Wireless (BTv4) \\ \hline
    \end{tabularx}
    \label{your_label_here}
\end{table}

\begin{itemize}
    \item Sport monitorozás
    \begin{itemize}
        \item Órák
        \item Sport
        \item Extrém körülmények
        \item Prevenció (hirtelen szívhalál)
    \end{itemize}
\end{itemize}

\subsection{Szívritmusszabályzók, pacemakerek}
\begin{itemize}
    \item A mellkasban vagy a hasban műtét során elhelyezett kis eszköz
    (szenzor és aktuátor) , ami, elektromos impulzusok segítségével
    képes a szív ingerképző és ingerületvezető feladatait átvenni,
    illetve kontrollálni.
    \item Aritmia
    \begin{itemize}
        \item szívritmuszavar
        \item aritmia esetén a szív vagy túl lassan (bradikardia), vagy túl gyorsan
        (tachikardia), vagy szabálytalanul veri a testbe a vért
        \item ez olyan tüneteket okozhat, mint fáradtság, légszomj, ájulás
        \item súlyosabb esetben azonban roncsolhatja a szervezetet,
        eszméletvesztéshez, halálhoz is vezethet
    \end{itemize}
    \item Pacemaker beültetésével ezek a tünetek eltűnhetnek, a betegek
    teljes, aktív életet élhetnek.
\end{itemize}

\subsection{Kinek van szüksége pacemakerre?}
\begin{itemize}
    \item Általában az orvosok az előzőekben vázolt esetekben
    ajánlanak pacemakert.
    \begin{itemize}
        \item A leggyakoribb ok a bradikardia.
    \end{itemize}
    \item További esetek lehetnek
    \begin{itemize}
        \item Öregedés vagy szívbetegség okozta szinuszcsomó problémák
        miatt
        \item Pitvarfibrilláció esetén (öngerjesztő hatás, az aritmia egy fajtája)
        \item Bizonyos gyógyszerek szedése mellett, pl. Béta-blokkolók
        esetében a szívverés ritmusa lelassulhat.
        \item Szívizom problémák esetén
        \item Hosszú QT szindróma esetén
    \end{itemize}
\end{itemize}

\subsection{Hogyan működik a pacemaker?}
\begin{itemize}
    \item Egységei
    \begin{itemize}
        \item elem,
        \item számítógéppel ellátott generátor,
        \item vezeték + szenzor $\rightarrow$ elektróda
    \end{itemize}
    \item Mit csinál?
    \begin{itemize}
        \item Az elektródák detektálják a szív
        elektromos aktivitását, és adatokat
        küldenek a számítógépesített
        generátorba
        \item Helytelen szívritmus esetén a számítógép
        arra utasítja a generátort, hogy
        elektromos pulzust küldjön a szívnek
    \end{itemize}
\end{itemize}

\subsection{Pacemakertípusok}
\begin{itemize}
    \item Egyelektródás pacemaker
    \item Pitvar-kamrai pacemaker
    \item Biventricularis pacemaker
\end{itemize}

\subsection{Egyelektródás pacemaker}
\begin{itemize}
    \item A vezeték a jobb
    kamra vagy a jobb
    pitvar és a generátor
    között szállít
    impulzusokat.
\end{itemize}

\subsection{Pitvar-kamrai pacemaker}
\begin{itemize}
    \item A vezetékek a jobb kamra, a jobb pitvar és a
    generátor között szállítanak impulzusokat.
    \item Segíti a két kamra
    összehúzódásának
    időzítését
\end{itemize}

\subsection{Biventricularis pacemaker}
\begin{itemize}
    \item A vezetékek a pitvar, mind a két
    kamra és a generátor között
    szállítanak impulzusokat
    \item Az előző típus a jobb pitvar és a
    jobb kamra együttműködését
    segítette
    \item Ez a típus egy harmadik
    vezetékkel a két kamra egyidejű összehúzódását
    segíti
\end{itemize}

\subsection{A beültetésről}
\begin{itemize}
    \item nagyjából 1 órás
    \item kulcscsont alatti területet helyileg
    érzéstelenítik, majd kis bevágás a bőrön
    \item az elektródát egy vénán keresztül bevezetik a
    szívbe $\rightarrow$ ellenőrzik (pl.: rtg-n) a helyes
    elhelyezkedést
    \item ezután csatlakoztatják a pacemakerhez, majd
    ezt a kulcscsont alatt képzett kis üregbe
    helyezik
\end{itemize}

\subsection{A szív elektrofiziológiás vizsgálata diagnosztikai katéterrel}
\begin{itemize}
    \item Speciális katéterek
    \item Egyszerre több ponton is lehet mérni
    \item Rtg-vel, illetve ultrahanggal támogatott
    elhelyezés
\end{itemize}